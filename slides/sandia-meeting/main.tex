\documentclass[xcolor=dvipsnames]{beamer}
\usepackage[english]{babel}
\usepackage{hyperref}

\usepackage{amsmath}
\usepackage{amssymb}
\usepackage{xspace}
\usepackage{ltl} 
\usepackage{multirow}
\usepackage{calc}
\usepackage{tikz}
\usetikzlibrary{arrows,positioning,backgrounds,calc,shapes,decorations,automata,snakes}
\usetikzlibrary{mindmap}

\usefonttheme{professionalfonts}
\usepackage{appendixnumberbeamer}

\setbeamertemplate{sections/subsections in toc}[sections numbered]
\setbeamercolor{section in toc}{parent=normal text}
\setbeamertemplate{section in toc shaded}[default][60]
\setbeamertemplate{navigation symbols}{}

\usepackage{tcolorbox}

\usepackage{media9}
\usepackage{multimedia}

\makeatletter
%\setbeamertemplate{footline}[text line]{\strut\hfill\insertframenumber/\inserttotalframenumber\hspace*{1em}\hskip-\beamer@leftmargin}
\makeatother


\usepackage{amsmath} 
\usepackage{amssymb} 


\usepackage{ltl} 
\usepackage{multirow}
\usepackage[vlined]{algorithm2e}
\usepackage{mathtools}

\usepackage{color}
\usepackage{subfig}
\usepackage{tikz}
\usepackage{hyperref}
\usetikzlibrary{arrows,fit,shapes,automata}
\usetikzlibrary{positioning,fit,calc,shapes}
\usetikzlibrary{decorations.fractals}
\usetikzlibrary{decorations.markings}
\usepackage{stmaryrd}

\newcommand{\Rayna}[1]{{\textcolor{magenta}{ \textbf{Rayna:} #1 $\spadesuit$ }}}
\newcommand{\Suda}[1]{{\textcolor{blue}{ \textbf{Suda:} #1 $\spadesuit$ }}}
\newcommand{\Ufuk}[1]{{\textcolor{red}{ \textbf{Ufuk:} #1 $\spadesuit$ }}}
\newcommand{\todo}[1]{{\textcolor{red}{TODO:} #1}}
\newcommand{\comment}[1]{}

\newtheorem{example}{Example}
\newtheorem{theorem}{Theorem}
\newtheorem{corollary}{Corollary}
\newtheorem{proposition}{Proposition}
\newcommand*{\qed}{\hfill\ensuremath{\blacksquare}}

\newcommand{\init}{\mathsf{init}}
\newcommand{\belief}{\mathsf{belief}}
\newcommand{\abstr}{\mathsf{abstract}}
\newcommand{\vis}{\mathit{vis}}
\newcommand{\Vis}{\mathit{Vis}}
\newcommand{\succs}{\mathit{succ}}
\newcommand{\beliefs}{\mathcal{P}(L_t)}
\newcommand{\subbeliefs}{\mathcal{P}(\tilde{L}^i_t)}
\newcommand{\Surveillance}{\mathsf{Surveillance}}
\newcommand{\beliefF}{\mathit{belief}}

\newcommand{\states}{S}
\newcommand{\trans}{T}
\newcommand{\part}{\mathcal{Q}}

\newcommand{\post}{\mathit{succ}_t}

\newcommand{\outcome}{\mathit{outcome}}
\newcommand{\counterex}{\mathcal{C}}

\newcommand{\bools}{\mathbb{B}}
\newcommand{\true}{\mathit{true}}
\newcommand{\false}{\mathit{false}}
\newcommand{\nats}{\mathbb{N}}

\newcommand{\SP}{\mathcal{SP}}
\newcommand{\AP}{\mathcal{AP}}

\newcommand{\locspec}{\mathit{local}}





   

\begin{document}

\begin{frame}
\frametitle{Synthesis of surveillance strategies  {\small [submitted to ICRA'18]} \newline \small S.\ Bharadwaj, R. Dimitrova, and U. Topcu}

 Temporal evolution of uncertainty as a synthesis objective

\bigskip

\input{grid.tex}

\uncover<4>{
\bigskip
\structure{Challenge:} Exponential size of the set of uncertainty sets

\bigskip
\structure{Approach:} Abstraction-based surveillance strategy synthesis
}
\end{frame}

\begin{frame}
\frametitle{Safety surveillance objectives}


\structure{Surveillance :} the uncertainty should never grow above 10 cells

\smallskip

\structure{Task specification:} visit the goal location infinitely often


\bigskip

 \begin{center} 
\movie[
  height = 4cm,
  width = 7cm,
  showcontrols,
  poster,
  borderwidth=1pt
] 
{}{video-safety.mp4}

 \end{center}

\end{frame}

\begin{frame}
\frametitle{Liveness surveillance objectives}


\structure{Surveillance :} infinitely often know precisely the target's location

\smallskip

\structure{Task specification:} visit the goal location infinitely often


\bigskip

 \begin{center} 
\movie[
  height = 4cm,
  width = 7cm,
  showcontrols,
  poster,
  borderwidth=1pt
] 
{}{video-liveness.mp4}

 \end{center}

\end{frame}

\end{document}
