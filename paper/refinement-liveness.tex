\subsubsection{Forward belief-set propagation}

To check if an abstract counterexample graph $\counterex_\abstr$ is concretizable, we construct a finite graph $\mathcal{D}$ whose nodes are labelled with elements of $\states_\belief$ and with nodes of $\counterex_\abstr$.
By construction we will ensure that if a node $d$ in $\mathcal D$ is labelled with a belief state $(l_a,B_t)$, and with a node $v$, then $v$ is labelled wit $(l_a,A_t)$ in $\counterex_\belief$, where $B_t \subseteq \gamma(A_t)$. 

Initially $\mathcal D$ contains a single node $d_0$ labelled with $\{s_\belief^\init\}$ and with the initial node $v_0$ of $\counterex_\abstr$. Consider a node $d$ in $\mathcal D$ labelled with a belief state $(l_a,B_t)$ and with a node $v$ in $\counterex_\belief$. For every child $v'$ of $v$, labelled with some abstract state $(l_a',A_t')$ we proceed as follows. We let ${B_t}' = \post(B_t) \cap \gamma(A_t')$. If there exists a node $d'$ in $\mathcal D$ labelled with $(l_a',B_t')$ and with $v'$, then we add an edge from $d$ to $d'$ in $\mathcal{D}$. Otherwise, we create such a node and add the edge. We continue until no more nodes and edges can be added to $\mathcal D$. The procedure is guaranteed to terminate, since both the graph $\counterex_\belief$ and the set $\states_\belief$ are finite, and we add a node labelled with $(v,s_\belief)$ to $\mathcal D$ at most once.

If the graph $\mathcal D$ contains a reachable cycle (it suffices to consider simple cycles, i.e., without repeating intermediate nodes) $\rho = d_0,\ldots,d_n$ with $d_0 = d_n$ such that some $d_i$ is labelled with $(l_a,B_t)$ where $(l_a,B_t) \models p_k$, then we conclude that the abstract counterexample $\counterex_\abstr$ is not concretizable. If no such cycle exists, then $\mathcal D$ is a concrete counterexample graph for the belief game $(G_\belief,\LTLglobally\LTLfinally p_k)$. 


\begin{theorem}\todo{correctness of counterexample analysis}
\end{theorem}

\subsubsection{Backward partition splitting}

Consider a path in the graph $\mathcal{D}$ of the form $\pi = d_0,\ldots, d_n,d_0',\ldots,d_m'$ where $d_n = d_m'$, and where for some $0 \leq i \leq m$ for the label $(l_a^,B_t^i)$ it holds that $(l_a^,B_t^i) \models p_k$. Let 
$\pi_\abstr = v_0,\ldots, v_n,v_0',\ldots,v_m'$ be the sequence of nodes in $\counterex_\abstr$ corresponding to the labels in $\pi$. By construction of $\mathcal D$, $\pi_\abstr$ is a path in $\counterex_\abstr$ and $v_n = v_m'$. We apply the refinement procedure from the previous section to the whole path $\pi_\abstr$, as well as to the to the path-prefix $v_0,\ldots, v_n$.

\begin{theorem}\todo{progress property of the refinement procedure: counterexample eliminated}
\end{theorem}

\begin{example}
\todo{illustrating example}
\end{example}
