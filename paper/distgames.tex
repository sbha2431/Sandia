In the sequel, to simplify notation we assume that $L_1 = L_2 = \dots = L_t \triangleq L$  in the surveillance game structure, i.e, all $n$ sensors and the target operate in the same state space. We denote this as a $G$ defined over set $L$. For what follows, let $G=(\states,s^\init,\trans,\vis_1,\ldots,\vis_n,\Vis)$ be a multi-agent surveillance game structure with $n$ sensors defined over $L$.

\subsection{State partition}
Given a multi-agent surveillance game $G$, we define a \emph{state partition} of size $n$ as a set of subsets $\widetilde{L} = \{\widetilde{L}_i \mid \bigcup_i^n L_i = L \}$. 


\subsection{Surveillance subgames}
We now describe how to construct a tuple of single-agent surveillance game structures $(G^{1}_{\tilde{S}},\ldots,G^{n}_{\tilde{S}})$ that contains one  \emph{surveillance subgame} $G^{i}_{\tilde{S}}$ for each mobile sensor  $i$. The construction is defined over a state partition $\widetilde{L}$ with each subgame, $G^{i}_{\tilde{S}}$, defined over the subset $\widetilde{L}_i$. Since, the target and sensor operate on the same state space we will have $\widetilde{L}^i_s = \widetilde{L}^i_t = \widetilde{L}_i$. Additionally, to each $\widetilde{L}^i_t$ we add an \emph{auxiliary location} $k_i$ that encapsulates all possible locations of the target that are outside of this subgame's region, i.e., all locations in $L \setminus \widetilde{L}^i_t$.  We then model transitions leaving or entering $\widetilde{L}^i_t$ as transitions to or from location $k_i$ respectively.


 %parametrised by a subset of sensor locations $\widetilde{L}_i \subseteq L_i$ and a subset of target locations $\widetilde{L}^i_t \subseteq L_t$ for each sensor $i$. These sets of locations form the state space $\widetilde{S}_i$ of the subgame $G^{i}_{\hat{S}}$.

%This ensures that all transitions in the subgame stay in the subgame. 

More formally, given partition $\widetilde{L}_i \subseteq L$ and we define the \emph{subgame} of $G$ corresponding to sensor $i$ as the tuple $G^{i}_{\tilde{S}} = (\widetilde{S}_i,\widetilde{s}_i^{init},\widetilde{T}_i,\widetilde{\vis}_i)$  where
\begin{itemize}
\item $\widetilde{S}_i= \widetilde{L}_i \times (\widetilde{L}^i_t \cup k_i)$ is the set of states.
\item The set $\widetilde{T}_i$ consists of two types of transitions: the transitions in $T{\downarrow } i$ that originate and end in locations part of the subgame's region are preserved as they are, while the transitions of the target exiting or entering $\widetilde{L}^i_t$ are replaced by transitions to and from location $\hat{l}^i_t$ respectively, since $k_i$ represents all target locations outside of  $\widetilde{L}^i_t$. 
Formally, for every pair of states $(\widetilde{l}_i,\widetilde{l}_t) \in \widetilde{S}_i$ and $(\widetilde{l}_i',\widetilde{l}_t') \in \widetilde{S}_i$ we have that $((\widetilde{l}_i,\widetilde l_t),(\widetilde{l}_i',\widetilde l_t')) \in \widetilde T_i$ if and only if there exists a transition
 $((\widetilde{l}_i,l_t),(\widetilde{l}_i',l_t')) \in T{\downarrow}i$ for which the following conditions are satisfied:
 \begin{itemize}
 \item if $\widetilde l_t \in \widetilde L_t ^i$ and $\widetilde l_t' \in \widetilde L_t ^i$, then 
 $\widetilde l_t = l_t$ and $\widetilde l_t'= l_t'$, that is, we have a \emph{transition internal for the region $\widetilde L_t^i$};
 \item if $\widetilde l_t \in \widetilde L_t ^i$ and $\widetilde l_t' =  \hat l^i_t$, then 
 $l_t \in \widetilde L_t^i$ and $l_t' \not\in \widetilde L_t^i$, that is, we have a \emph{transition exiting the region $\widetilde L_t^i$}; 
 \item if $\widetilde l_t= \hat l^i_t$ and $\widetilde l_t' \in  \widetilde L_t ^i$, then 
 $l_t \not \in \widetilde L_t^i$ and $l_t' \in \widetilde L_t^i$, that is, we have a \emph{transition entering the region $\widetilde L_t^i$}; 
 \item if $\widetilde l_t= \hat l^i_t$ and $\widetilde l_t' =  \hat l^i_t$, then 
 $l_t \not \in \widetilde L_t^i$ and $l_t' \not\in \widetilde L_t^i$, that is, we have a \emph{transition completely outside $\widetilde L_t^i$}.
\end{itemize}  

  \item The visibility function $\widetilde{\vis}_i$ in the subgame $G^{i}_{\hat{S}}$ agrees with the visibiity function $\vis_i$ of sensor $i$ in the original game when the target's location is in the subgame's region. Target locations outside of the region $\widetilde L_t^i$ are invisible to the sensor in the subgame. Formally, 
 \[\widetilde{\vis}_i(\widetilde{l}_i,l_t) = \begin{cases}
\vis_i(\widetilde{l}_i,l_t) & l_t \in \widetilde{L}_t^i, \\
\false & l_t \notin \widetilde{L}_t^i.
\end{cases}
\]
\end{itemize}
Note that in this construction, sensor $i$ is not able to leave the region of locations $\widetilde{L}_i$. Furthermore, all the information about the target's behaviour outside of  the subgame's region is completely hidden from the mobile sensor controller, since all locations outside of  $\widetilde{L}_t^i$ are represented by the single location $\hat{l}_t^i$.
Finally we remark that the previously defined notions of $\succs_t(\widetilde{l}_i,\widetilde{l}^i_{t})$, $\succs_t(\widetilde{l}_i,\widetilde{L}^i_{t})$, and $\succs(\widetilde{l}_i,\widetilde{l}^{t},\widetilde{l}_{t}')$ for the global game structure $G$ follow analogously in this construction for $\widetilde{T}$ which we denote as $\widetilde{\succs}$.

\subsection{Belief subgame}
\Rayna{I think we should define in the previous section single-agent surveillance game structures as a special case of multi-agent surveillance game structures. Then, since a surveillance subgame is essentially a single-agent surveillance game, and since there is nothing specific in the definition below, we can remove this subsection and  move the last remark to the next subsection.}
Given a surveillance subgame $G^{i}_{\tilde{S}} = (\widetilde{S}_i,\widetilde{s}_i^{init},\widetilde{T}_i,\widetilde{\vis}_i)$, we can construct a belief subgame $G^{i}_{\widetilde{S}_\belief} = (\widetilde{\states}_{i_\belief},\widetilde{s}^\init_{i_\belief},\widetilde{\trans}_{i_\belief})$ 

\begin{itemize}
\item $\states_{i_\belief} = \widetilde{L}_i \times \mathcal{P}(\widetilde{L}^i_t \cup k_i)$ is the set of states,
\item $\widetilde{\trans}_{i_\belief} \subseteq \widetilde{\states}_{i_\belief} \times \widetilde{\states}_{i_\belief}$ is the transition relation where $((\widetilde{l}_i, B_t),(\widetilde{l}_i', B_t')) \in \widetilde{\trans}_{i_\belief}$ iff $\widetilde{l}_i' \in  \widetilde{\succs}(\widetilde{l}_i,l_t,l_t')$ for some $l_t \in B_t$ and $l_t' \in B_t'$ and one of these holds:
\begin{itemize}
\item[(1)] $B_t' = \{l_t'\}$, $l_t' \in \widetilde{\post}(\widetilde{l}_i,B_t)$, $\widetilde{\vis}_i(l,l_t') = \true$;
\item[(2)] $B_t' = \{l_t' \in \widetilde{\post}(\widetilde{l}_i,B_t)  \mid  \widetilde{\vis}_i(l,l_t') = \false \}$.
%\item[(1)] $B_t' = \{l_t'\}$ for some $l_t'$ such that $\vis(l_a,l_t') = \true$ and
%there exists $l_t \in B_t$ with $((l_a,l_t),(l_a',l_t')) \in \trans$;
%\item[(2)] $\begin{array}{lll}
%B_t' = \{l_t' & \mid & \vis(l_a,l_t') = \false \text{ and } \\
%&& \exists l_t \in B_t.\ ((l_a,l_t),(l_a',l_t')) \in \trans\}. 
%\end{array}
%$
\end{itemize}
\end{itemize}

 For brevity, we omit the definitions of $\widetilde{\succs}_t$, $\widetilde{\succs}$, runs and strategies for the belief subgames as the definitions follow trivially from those for the belief games but in the subspace of transitions and states as defined above.

The outcome of given strategies $f_{i}$ and $f_{t_i}$ for sensor $i$ and the target in the subgame, $\outcome(G^{i}_{\widetilde{S}_\belief},f_s,f_t)$, is a run $s_0,s_1,\ldots$ of $G^{i}_{\widetilde{S}_\belief}$ such that for every $k \geq 0$, we have $s_{k+1} = f_i(s_0,\ldots,s_k,B^k_{t_i})$, where $B^k_{t_i} = f_{t_i}(s_0,\ldots,s_k)$.

\paragraph*{Remark}$B^k_{t_i}$ is the belief that sensor $i$ holds on the position of the target. Since the subgame for each sensor will be executing concurrently, the global belief on the location of the target will be a combination of the knowledge of the individual sensors. This is discussed in more detail in the sequel.

\subsection{Composition of subgames}
Given $n$ belief subgames $(G^{1}_{\tilde{S}_\belief},\ldots,G^{n}_{\tilde{S}_\belief})$ over state partition $\widetilde{L}$, we define a \emph{run} in the composed belief game $G_\belief$ as an infinite sequence $s_0,s_1,\ldots$ of states where $s_k = (\tilde{s}_{1_k} \ldots \tilde{s}_{n_k})$ is the combined state of all the subgames at timestep $k$, and  $(\tilde{s}_{i_k},\tilde{s}_{{i+1}_k}) \in \tilde{T}_{i_\belief}$ for all $k$. 

Given a strategy for each sensor $f_{s_i}$ in the corresponding subgames, we compose the subgame strategies into a global strategy in the same way. 


\subsection{Global belief vs local belief}

Given $n$ belief subgames, at timestep $k$, sensor $i$ holds belief $B^k_{t_i}$. This is the \emph{local} belief as it is held by a single sensor and is not shared with the others. We first present the following fact: \todo{\textbf{not sure if this is a theorem, but not sure what else to call it.}}
\begin{theorem} 
At timestep $k$, if there exists $i\in \{1\dots n\}$ such that $B^k_{t_i} = \{l_t\}$ for $l_t \in \widetilde{L}_i^t$, then for all $j \neq i$, we have $\hat{l}^t_j \in B^k_{t_j}$.
\end{theorem}
\begin{proof}
By construction of the subgames \textbf{more to come}
\end{proof}
\begin{corollary}\label{corr:uniqi}
There can exist at most one $i\in \{1\dots n\}$ such that $B|^k_{t_i}| = 1$ and $\hat{l}^t_j \in B^k_{t_j} \notin B|^k_{t_i}|$.
\end{corollary}
Intuitively, this states that if the target is in vision of sensor $i$, all the other sensors must have $\hat{l}^t_j$ in their belief sets. Recall that $\hat{l}^t_j$ indicates that the target is not in the state space of the subgame corresponding to sensor $j$. The corollary states that if the target is in vision of sensor $i$, then no other sensor can claim to see the same target. 
The global belief is defined as:
\[B^k_t \triangleq 
\bigcap_{i}^n B^k_{t_i}
\]

 




\subsection{Distributed surveillance synthesis problem}
Given a global surveillance game $(G,\varphi)$ with $n$ sensors, compute a winning strategy for each sensor subgame $G^{i}_{\widetilde{S}_\belief}$ with local surveillance objective $\varphi_i$ such that the composition of the $n$ winning policies guarantees that the global surveillance objective $\varphi$ is satisfied. Formally, compute the winning strategy $f_i$ for the sensor in each belief subgame such that if $\outcome(G^{i}_{\widetilde{S}_\belief},f_{s_i},f_{t_i}) \models \varphi_i$, then the composed strategy for all agents $f_s = f_{s_1} \bigoplus f_{s_2} \dots \bigoplus f_{s_n}$ must be winning for the global property, i.e, $\outcome(G_{\belief},f_{s},f_{t}) \models \varphi$

\Rayna{The definition above mixes up several things (1) the distributed surveillance synthesis problem where each sensor has only a local objective (2) the reduction from a given multi-agent surveillance synthesis problem (satisfying some restriction) to a distributed surveillance synthesis problem with local objectives such that  the composition of the resulting strategies satisfies the global objectives. This property of the reduction is something we have to prove.}




