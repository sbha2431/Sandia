\subsection{Surveillance subgames}
Given a surveillance game structure $G$ with $i$ sensors where $G = (\states,s^\init,\trans,\vis)$, we construct a \emph{surveillance subgame} corresponding to each sensor $i$. To do this, we select a subset of states $\widetilde{L}_i \subseteq L_i$ and $\widetilde{L}^i_t \subseteq L_t$ which forms the state space $\widetilde{S}_i$ of the subgame. We append a state $\hat{l}^i_t$ to $\widetilde{L}^i_t$ that corresponds to when the target leaves the states $\widetilde{L}_t$ corresponding to the subgame. This ensures that all transitions in the subgame stay in the subgame. More formally, we say $G^{i}_{\hat{S}} = (\widetilde{S}_i,\widetilde{s}_i^{init},\widetilde{T}_i,\widetilde{\vis}_i)$ is a \emph{subgame} of $G$ corresponding to sensor $i$ where 

\begin{itemize}
\item $\widetilde{S}_i= \widetilde{L}_i \times (\widetilde{L}^i_t \cup \hat{l}^i_t)$ where $\widetilde{L}_i\subseteq L_i$, and $\widetilde{L}_t \subseteq L$. 
\item Informally, any transitions in in $T$ that involve the target staying in $\widetilde{L}^i_t$ are kept. Those transitions that involve either entering or exiting $\widetilde{L}^i_t$ are replaced with $\hat{l}^i_t$ which is the state representing all states in $L_t$ not in $\widetilde{L}^i_t$. Formally, the transitions $\widetilde{T}$ are defined as a union of the following sets \todo{\textbf{Is there a more compact way of defining this?}} For all $\tilde{l_i} \in \widetilde{L}_i$, and $((\tilde{l}_i,l_t),(\tilde{l}_i',l_t')) \in T$, \[\widetilde{T}_i \triangleq \begin{cases}
 \{((\tilde{l}_i,l_t),(\tilde{l}_i',l_t')) \mid l_t,l_t' \in \widetilde{L}^i_t \} \; \cup \\
 \{((l_i,l_t),(l_i',\hat{l}^i_t)) \} \mid l_t' \notin \widetilde{L}_t^i, l_t \in \widetilde{L}_t^i \}  \; \cup \\
 \{((l_i,\hat{l}^i_t),(l_i',\hat{l}^i_t)) \mid l_t,l_t' \notin \widetilde{L}_t^i \}  \; \cup \\
 \{((l_i,\hat{l}^i_t),(l_i',l_t')) \mid l_t \notin \widetilde{L}_t^i, l_t' \in \widetilde{L}_t^i \} 
  \end{cases}   \]
  
\item Informally, we state the visibility function for the subgame $\widetilde{\vis}_i$ must follow the values of the visibiity function of the corresponding sensor $\vis_i$ except in the case that the target is outside of the states of the subgame, i.e, it is in state $\hat{l}^t_i$. Formally,  \[\widetilde{\vis}_i(\tilde{l}_i,l_t) = \begin{cases}
\vis_i(\tilde{l}_i,l_t) & l_t \in \widetilde{L}_t^i \\
\false & l_t \notin \widetilde{L}_t^i
\end{cases}
\]
\end{itemize}
Note that in this construction, sensor $i$ is not able to leave the subset of states $\tilde{L}_i$ and all information of the target when it is not in $\tilde{L}_t^i$ is hidden as the state $\hat{l}_t^i$ represents any state not in $\tilde{L}_t^i$. Finally we remark that the previously defined notions of $\succs_t(\tilde{l}_i,\tilde{l}^i_{t})$, $\succs_t(\tilde{l}_i,\tilde{L}^i_{t})$, and $\succs(\tilde{l}_i,\tilde{l}^{t},\tilde{l}_{t}')$ for the global game structure $G$ follow analogously in this construction for $\widetilde{T}$ which we denote as $\widetilde{\succs}$.

\subsection{Belief subgame}
Given a surveillance subgame $G^{i}_{\hat{S}} = (\widetilde{S}_i,\widetilde{s}_i^{init},\widetilde{T},\widetilde{\vis}_i)$, we can construct a belief subgame $G^{i}_{\tilde{S}_\belief} = (\widetilde{\states}_{i_\belief},\tilde{s}^\init_{i_\belief},\tilde{\trans}_{i_\belief})$ 

\begin{itemize}
\item $\states_{i_\belief} = \tilde{L}_i \times \mathcal{P}(\tilde{L}^i_t \cup \hat{l}_t^i)$ is the set of states,
\item $\tilde{\trans}_{i_\belief} \subseteq \tilde{\states}_{i_\belief} \times \tilde{\states}_{i_\belief}$ is the transition relation where $((\tilde{l}_i, B_t),(\tilde{l}_i', B_t')) \in \tilde{\trans}_{i_\belief}$ iff $\tilde{l}_i' \in  \tilde{\succs}(\tilde{l}_i,l_t,l_t')$ for some $l_t \in B_t$ and $l_t' \in B_t'$ and one of these holds:
\begin{itemize}
\item[(1)] $B_t' = \{l_t'\}$, $l_t' \in \widetilde{\post}(\tilde{l}_i,B_t)$, $\widetilde{\vis}_i(l,l_t') = \true$;
\item[(2)] $B_t' = \{l_t' \in \widetilde{\post}(\tilde{l}_i,B_t)  \mid  \widetilde{\vis}_i(l,l_t') = \false \}$.
%\item[(1)] $B_t' = \{l_t'\}$ for some $l_t'$ such that $\vis(l_a,l_t') = \true$ and
%there exists $l_t \in B_t$ with $((l_a,l_t),(l_a',l_t')) \in \trans$;
%\item[(2)] $\begin{array}{lll}
%B_t' = \{l_t' & \mid & \vis(l_a,l_t') = \false \text{ and } \\
%&& \exists l_t \in B_t.\ ((l_a,l_t),(l_a',l_t')) \in \trans\}. 
%\end{array}
%$
\end{itemize}
\end{itemize}

 For brevity, we omit the definitions of $\widetilde{\succs}_t$, $\widetilde{\succs}$, runs and strategies for the belief subgames as the definitions follow trivially from those for the belief games but in the subspace of transitions and states as defined above.

The outcome of given strategies $f_{i}$ and $f_{t_i}$ for sensor $i$ and the target in the subgame, $\outcome(G^{i}_{\tilde{S}_\belief},f_s,f_t)$, is a run $s_0,s_1,\ldots$ of $G^{i}_{\tilde{S}_\belief}$ such that for every $k \geq 0$, we have $s_{k+1} = f_i(s_0,\ldots,s_k,B^k_{t_i})$, where $B^k_{t_i} = f_{t_i}(s_0,\ldots,s_k)$.

\paragraph*{Remark}$B^k_{t_i}$ is the belief that sensor $i$ holds on the position of the target. Since the subgame for each sensor will be executing concurrently, the global belief on the location of the target will be a combination of the knowledge of the individual sensors. This is discussed in more detail in the sequel.

\subsection{Global belief vs local belief}
Given $n$ belief subgames, at timestep $k$, sensor $i$ holds belief $B^k_{t_i}$. This is the \emph{local} belief as it is held by a single sensor and is not shared with the others. We first present the following fact: \todo{\textbf{not sure if this is a theorem, but not sure what else to call it.}}
\begin{theorem} 
At timestep $k$, if there exists $i\in \{1\dots n\}$ such that $B^k_{t_i} = \{l_t\}$ for $l_t \in \widetilde{L}_i^t$, then for all $j \neq i$, we have $\hat{l}^t_j \in B^k_{t_j}$.
\end{theorem}
\begin{proof}
By construction of the subgames \textbf{more to come}
\end{proof}
\begin{corollary}\label{corr:uniqi}
There can exist at most one $i\in \{1\dots n\}$ such that $B|^k_{t_i}| = 1$ and $\hat{l}^t_j \in B^k_{t_j} \notin B|^k_{t_i}|$.
\end{corollary}
Intuitively, this states that if the target is in vision of sensor $i$, all the other sensors must have $\hat{l}^t_j$ in their belief sets. Recall that $\hat{l}^t_j$ indicates that the target is not in the state space of the subgame corresponding to sensor $j$. The corollary states that if the target is in vision of sensor $i$, then no other sensor can claim to see the same target. 
The global belief is defined as one of two cases:
\[B^k_t \triangleq \begin{cases}
\bigcup_{i}B^k_{t_i} & \forall \, i \in \{1\dots n\} \;\; |B^k_{t_i}| > 1\\
l_t & \exists \; i \;\; |B^k_{t_i}| = \{l_t\} 
\end{cases}
\]
Note that from corollary \ref{corr:uniqi}, we ensure that the $i$ satisfying the second condition is unique and hence, if the target is in vision, the size of our global belief is 1. 
\subsection{Composition of subgame policies}
Given $n$ belief subgames each with a surveillance objective $\varphi_i$, and a strategy for the corresponding sensor $f_{s_i}$, we compose the subgame strategies into a global strategy in the following way.

 \todo{\textbf{fill in}}
 




\subsection{Distributed surveillance synthesis problem}
Given a global surveillance game $(G,\varphi)$ with $n$ sensors, compute a winning strategy for each sensor subgame $G^{i}_{\tilde{S}_\belief}$ with local surveillance objective $\varphi_i$ such that the composition of the $n$ winning policies guarantees that the global surveillance objective $\varphi$ is satisfied. Formally, compute the winning strategy $f_i$ for the sensor in each belief subgame such that if $\outcome(G^{i}_{\tilde{S}_\belief},f_{s_i},f_{t_i}) \models \varphi_i$, then the composed strategy for all agents $f_s = f_{s_1} \bigoplus f_{s_2} \dots \bigoplus f_{s_n}$ must be winning for the global property, i.e, $\outcome(G_{\belief},f_{s},f_{t}) \models \varphi$





