\documentclass[letterpaper, 10 pt, conference]{ieeeconf}  % Comment this line out if you need a4paper

%\documentclass[a4paper, 10pt, conference]{ieeeconf}      % Use this line for a4 paper

\IEEEoverridecommandlockouts                              % This command is only needed if 
                                                          % you want to use the \thanks command

\overrideIEEEmargins                                      % Needed to meet printer requirements.

%\renewcommand{\baselinestretch}{2}

% See the \addtolength command later in the file to balance the column lengths
% on the last page of the document
\pdfminorversion=4
\usepackage{amsmath} 
\usepackage{amssymb} 


\usepackage{ltl} 
\usepackage{multirow}
\usepackage[vlined]{algorithm2e}
\usepackage{mathtools}

\usepackage{color}
\usepackage{subfig}
\usepackage{tikz}
\usepackage{hyperref}
\usetikzlibrary{arrows,fit,shapes,automata}
\usetikzlibrary{positioning,fit,calc,shapes}
\usetikzlibrary{decorations.fractals}
\usetikzlibrary{decorations.markings}
\usepackage{stmaryrd}

\newcommand{\Rayna}[1]{{\textcolor{magenta}{ \textbf{Rayna:} #1 $\spadesuit$ }}}
\newcommand{\Suda}[1]{{\textcolor{blue}{ \textbf{Suda:} #1 $\spadesuit$ }}}
\newcommand{\Ufuk}[1]{{\textcolor{red}{ \textbf{Ufuk:} #1 $\spadesuit$ }}}
\newcommand{\todo}[1]{{\textcolor{red}{TODO:} #1}}
\newcommand{\comment}[1]{}

\newtheorem{example}{Example}
\newtheorem{theorem}{Theorem}
\newtheorem{corollary}{Corollary}
\newtheorem{proposition}{Proposition}
\newcommand*{\qed}{\hfill\ensuremath{\blacksquare}}

\newcommand{\init}{\mathsf{init}}
\newcommand{\belief}{\mathsf{belief}}
\newcommand{\abstr}{\mathsf{abstract}}
\newcommand{\vis}{\mathit{vis}}
\newcommand{\Vis}{\mathit{Vis}}
\newcommand{\succs}{\mathit{succ}}
\newcommand{\beliefs}{\mathcal{P}(L_t)}
\newcommand{\subbeliefs}{\mathcal{P}(\tilde{L}^i_t)}
\newcommand{\Surveillance}{\mathsf{Surveillance}}
\newcommand{\beliefF}{\mathit{belief}}

\newcommand{\states}{S}
\newcommand{\trans}{T}
\newcommand{\part}{\mathcal{Q}}

\newcommand{\post}{\mathit{succ}_t}

\newcommand{\outcome}{\mathit{outcome}}
\newcommand{\counterex}{\mathcal{C}}

\newcommand{\bools}{\mathbb{B}}
\newcommand{\true}{\mathit{true}}
\newcommand{\false}{\mathit{false}}
\newcommand{\nats}{\mathbb{N}}

\newcommand{\SP}{\mathcal{SP}}
\newcommand{\AP}{\mathcal{AP}}

\newcommand{\locspec}{\mathit{local}}





   

\title{\LARGE \bf Distributed Synthesis of Surveillance Strategies for Mobile Sensors}

\author{Suda Bharadwaj$^{1}$ and Rayna Dimitrova$^{2}$ and Ufuk Topcu$^{1}$% <-this % stops a space
%\thanks{*This work was not supported by any organization}% <-this % stops a space
\thanks{$^{1}$Suda Bharadwaj and Ufuk Topcu are with the University of Texas at Austin}%
\thanks{$^{2}$Rayna Dimitrova is with the University of Leicester, UK.}%
}



\begin{document}

\maketitle
\thispagestyle{empty}
\pagestyle{empty}

%%%%%%%%%%%%%%%%%%%%%%%%%%%%%%%%%%%%%%%%%%%%%%%%%%%%%%%%%%%%%%%%%%%%%%%%%%%%%%%%

\begin{abstract}
We study the problem of synthesizing strategies for a mobile sensor network to conduct surveillance in partnership with static alarm triggers. We formulate the problem as a multi-agent reactive synthesis problem with surveillance objectives specified as a temporal logic formula. In order to avoid the state space blow-up arising from a centralized strategy computation, we propose a method to \emph{decentralize} the surveillance strategy synthesis by decomposing the multi-agent game into subgames that can be solved independently. We also decompose the global surveillance specification into local specifications for each sensor and show that sensors satisfying their local surveillance specifications guarantees the sensor network as a whole will satisfy the global surveillance objective. Thus, our method is able to guarantee global surveillance properties in a mobile sensor network while synthesizing completely decentralized strategies with no need for communication between the sensors. We also present a case study where we demonstrate an application of decentralized surveillance strategy synthesis.
\end{abstract}

%%%%%%%%%%%%%%%%%%%%%%%%%%%%%%%%%%%%%%%%%%%%%%%%%%%%%%%%%%%%%%%%%%%%%%%%%%%%%%%%

\section{INTRODUCTION}
The importance of surveillance in our daily life has been constantly growing in the past couple of decades, and with that, also the need for more efficient and sophisticated mechanisms for surveillance. One of the major challenges comes from the need to perform surveillance in large and complex environments, where it is not always feasible or cost effective to have complete surveillance coverage of the entire area at all times. Furthermore, sensors  might not always be able to classify threats, and often require human intervention to assess the threat level. It can thus be necessary to deploy multiple mobile sensors, that work together with conventional static sensors to maintain a sufficient level of knowledge on the location of a potential threat. This is particularly crucial in applications where it is necessary to monitor a potential threat which can move over a large area until it can be accounted for.

In a formal setting, designing a surveillance strategy for a (mobile) sensor network dealing with a potentially adversarial target can be modelled as a two-player game in which one player represents the sensor network and the other player represents the adversary. There are several variants of such games, including pursuit-evasion games~\cite{Chung2011} and graph-searching games~\cite{Kreutzer11}. In such games,	 the problem is formulated as enforcing eventual detection, which is, in essence, a search problem -- once the target is detected, the game ends. These types of games are too restrictive for applications where the goal is not to capture, but instead  to maintain information about the location of the adversary for an unbounded time horizon.

Another class of games used in physical security are \emph{Stackelberg} games, also known as leader-follower games. In such games the defender acts first, for example by placing their defence system, and the attacker follows with his action, possibly after obtaining information about the placed defence system. In recent years Stackelberg games have seen use in, among others, LAX airport,~\cite{jain2012overview} and the US Coast Guard~\cite{An11}. These games aim to compute randomized policies for the defender to protect target locations from an attacker. Extensions of this model~\cite{Basilico12} have been proposed to generate infinite-horizon patrolling strategies either for mobile resources alone or in concert with static alarm triggers~\cite{basilico2016security,Munoz13}. However, these models cannot be used to reason about the uncertain set of possible  locations of dynamic threats.

Our objective in this work  is not to just compute a patrolling strategy, but also to quantify the sensor network's knowledge of the possible locations of active threats and use this information to synthesize strategies for the mobile sensors that provide knowledge guarantees on the threat location over an infinite-time horizon.

As a motivating case study in this paper, we consider the use of autonomous drones working in cooperation with static sensors in wildlife conservation. UAVs are increasingly being adopted for monitoring of illegal hunting and poaching~\cite{schiffman2014drones}, though they are mostly remotely controlled~\cite{mulero2014remotely}. In Kenya, for example, remotely controlled drones were deployed in 2014 in an attempt to reduce poaching  by providing constant surveillance~\cite{Kenya}, allowing authorities to arrest rhino poachers when they are sensed by the drones. Autonomous UAVs  have not been used in this setting yet, and proposed plans involve drones following pre-programmed paths~\cite{Koh12}. In this paper, we propose a method for automatically constructing  \emph{autonomous reactive surveillance} strategies for multiple mobile sensors (like UAVs) working in concert with static sensors in the field. 

We study the problem of synthesizing strategies for enforcing \emph{temporal surveillance objectives}, such as the requirement to never let the sensor network's uncertainty about the target's location exceed a given threshold, or recapturing the target every time it escapes. To this end, we consider surveillance objectives specified in linear temporal logic (LTL), equipped with basic surveillance predicates. Our computational model is that of a two-player game played on a finite graph, whose nodes represent the joint possible locations of all the mobile sensors and the target, and whose edges model the possible (deterministic) moves between locations. The mobile sensors play the game with partial information, as they can only observe the target when  it is in the area of sight of one of the sensors. The target, on the other hand, always has full information about the location of the entire sensor network. In that way, we consider a model with one-sided partial information, making the computed strategy for the agent robust against a potentially more powerful adversary.


We formulate surveillance strategy synthesis as the problem of computing a joint winning strategy for the multiple mobile sensors in a partial-information game with a surveillance objective. Partial-information games with LTL objectives have been well studied~\cite{DoyenR11,Chatterjee2013} and it is well known that the synthesis problem is EXPTIME-hard~\cite{Reif84,BerwangerD08}. In a companion submission to CDC 2018 we describe a framework for formalizing \emph{single-agent} surveillance synthesis as a two-player game with partial information, and propose an abstraction-based method for solving such games. The interested reader is referred to the expended version~\cite{arxiv} for details about the abstraction-based synthesis method. The price of resorting to abstraction is the potential overapproximating of the set of possible target locations (that is, loss of precision in the sensors' knowledge) which may make satisfying surveillance requirements difficult. There is thus a trade-off between strictness of surveillance requirements, i.e, how closely a target needs to be tracked, and the size of the abstract game necessary for a surveillance strategy to exist. 

Sensor networks with a large number of dynamic sensors as well as static sensors can achieve better coverage, and thus, in general, can make do with much coarser abstractions to satisfy a given surveillance objective. However, even when using abstraction, the size of the game is exponential in the number of sensors. To address the blow-up of the state space incurred by large number of sensors, we propose a \emph{decentralized} synthesis method that aims to compute a surveillance strategy for each mobile sensor separately. Our main contribution is as follows:

 We decompose the original surveillance game into a set of subgames, one for each sensor. Accordingly, the global surveillance objective is  broken up into a local objective for each subgame. Our reduction guarantees, that if the local strategy in each subgame satisfies the local surveillance objective, then the composition of the strategies fulfills the global surveillance objective. This allows us to solve each subgame under its local surveillance objective independently, using off-the-shelf reactive synthesis tools. 

%Synthesis from LTL specifications~\cite{Pnueli1989}, especially from formulae in the efficient GR(1) fragment~\cite{Piterman2006}, has been extensively used in robotic planning (e.g.~\cite{wong2012,Kress2007}).
 There has been work in decentralized synthesis for GR(1) specifications, however, the synthesis process often involves a centralized computation as in \cite{Kloetzer06} or synchronization \cite{Salar17,Kloetzer11}. Our approach, on the other hand is fully  decentralized and the sensors require no communication as simply satisfying their local properties guarantees the global objective.

%, but surveillance-type objectives, such as the ones we study here, have not been considered so far. Epistemic logic specifications~\cite{MeydenV98} can refer to the knowledge of the agent about the truth-value of logical formulas, but, contrary to our surveillance specifications, are not capable of expressing requirements on the size of the agent's uncertainty.






\section{MOTIVATING CASE STUDY}\label{sec:casestudy}
We first describe the multi-agent surveillance synthesis problem informally, in the context of a motivating case study. 

We consider wildlife conservation in Africa, in particular, at the Selous Game Reserve (SGR) located in Tanzania, where the African Black Rhinoceros population is under serious threat due to poaching. We are motivated by a recommended anti-poaching initiative in the SGR by the World Heritage Centre,  to study the use of a sensor network for tracking the position of a potential poacher  with user-specified precision. Since the SGR is a very large area, the network consists of both mobile and static sensors. We apply the  distributed synthesis method proposed in this paper to synthesize surveillance strategies for the mobile sensors that satisfy the desired tracking requirement.


\begin{figure}
\centering
\subfloat[SGR interior landscape \cite{UN13} \label{fig:SGR-map}]{
\includegraphics[scale=0.15]{figs/SGR.png}
}\hspace{1cm}
%\hfill
\subfloat[Grid representation of the landscape in \ref{fig:SGR-map} \label{fig:SGR-grid}]{
\includegraphics[scale=0.15]{figs/SGR-grid-vis.png}
}

\caption{The landscape in \ref{fig:SGR-map} is coarsely represented as the gridworld in \ref{fig:SGR-grid}. The red regions represent impassable terrain. The yellow areas covered are by static sensors.}\vspace{-0.5cm}
\label{fig:casestudy}
\end{figure}

Figure \ref{fig:casestudy} shows a section of the SGR that we represent as a gridworld which will form the state space of the game. Each static sensor monitors a given area of the grid (shown in yellow) and detects any presence of the target (i.e., threat) in these states, but cannot determine the target's exact location. The  requirement is to ensure that over and over again, the set of potential locations of the target is reduced to $5$ cells. In other words, every time the target escapes from the vision of all  sensors, the network guarantees that eventually the uncertainty about its position will be reduced to $5$ grid cells.



%%%%%%%%%%%%%%%%%%%%%%%%%%%%%%%%%%%%%%%%%%%%%%%%%%%%%%%%%%%%%%%%%%%%%%%%%%%%%%%%

\section{GAMES WITH SURVEILLANCE OBJECTIVES}
\subsection{Surveillance Game Structures}
We define a \emph{surveillance game structure} to be  a tuple $G  = (\states,s^\init,T,\vis)$, where:
\begin{itemize}
\item $\states = L_a \times L_t$ is the set of states with $L_a$ the set of locations of the agent and $L_t$ the locations of the target;
\item $s^\init = (l_a^\init,l_t^\init)$ is the initial state;
\item $\trans \subseteq \states \times \states$ is the transition relation describing the possible moves of the agent and the target;
\item $\vis : \states \to \bools$ is a function that maps a state $(l_a,l_t)$ to $\true$ if and only if\emph{ $l_t$ is in the line of sight of $l_a$}.
\end{itemize}

\begin{example}
\todo{grid plus small part of corresponding game structure}
\end{example}
\subsection{Belief-Set Game Structures}

For a surveillance game structure $G  = (\states,s^\init,T,\vis)$ we define the corresponding \emph{belief game structure} $G_\belief  = (\states_\belief,s^\init_\belief,T_\belief)$ with the following components
\begin{itemize}
\item $\states_\belief = L_a \times \mathcal P(L_t)$ is the set of states with $L_a$ the set of locations of the agent, and $\mathcal P(L_t)$ the set of belief sets describing information about the location of the target;
\item $s^\init_\belief = (l_a^\init,\{l_t^\init\})$ is the initial state;
\item $\trans_\belief \subseteq \states_\belief \times \states_\belief$ is the transition relation such that $((l_a, B_t),(l_a', B_t')) \in T_\belief$ if and only if one of the following two conditions is satisfied:
\begin{itemize}
\item $B_t' = \{l_t'\}$ for some $l_t'$ such that $\vis(l_a,l_t') = \true$ and
there exists $l_t \in B_t$ with $((l_a,l_t),(l_a',l_t')) \in T$;
\item $\begin{array}{lll}
B_t' = \{l_t' & \mid & \vis(l_a,l_t') = \false \text{ and }\\
&&\exists l_t \in B_t.\; ((l_a,l_t),(l_a',l_t')) \in T\}.
\end{array}
$
\end{itemize}
The first condition captures the successor locations of the target that can be observed from the agent's current location, while the second corresponds to the belief set consisting of all possible successor locations of the target not visible from the agent's current location.
\end{itemize}

\noindent{\bf Remark} \todo{explain sequence of agent's observation and action}
\begin{example}
\todo{small part of corresponding belief-set game structure}
\end{example}


With each state $(l_a,B)$ in $\states_\belief$ we associate a set of states in $G$ defined as $\gamma((l_a,B)) = \{(l_a,l_t) \mid l_t \in B\}$.

\subsection{Temporal Quantitative Surveillance Objectives}

We consider a set of \emph{surveillance predicates} $\SP = \{p_k \mid k \in \nats\}$, where for $k \in \nats$ we say that a state $(l_a,B_t)$ in the belief game structure satisfies $p_k$ (denoted $(l_a,B_t) \models p_k$) iff 
$|\{l_t \in B_t \mid \vis(l_a,l_t)  = \false \}| \leq k$. Intuitively, $p_k$ is satisfied by the states in the belief game structure where the size of the belief set does not exceed the threshold $k \in \nats$.

We consider surveillance objectives expressed by formulas of linear temporal logic (LTL) over surveillance predicates. LTL surveillance formulas are generated by the grammar
\[\varphi := p \;\mid\; \neg \varphi \;\mid\; \varphi \vee \varphi \;\mid\; \LTLnext  \varphi  \;\mid\; \varphi \LTLuntil \varphi,\]

where $p \in \SP$ is a surveillance predicate, $\LTLnext$ is the \emph{next} operator, and $\LTLuntil$ is the \emph{until} operator. We also define the derived operators 
\emph{finally} $\LTLfinally \varphi = \true \LTLuntil \varphi$ and 
\emph{globally} $\LTLglobally \varphi = \neg \LTLfinally \neg \varphi$.
Of special interest will be formulas of the form $\LTLglobally p_k$, termed \emph{safety surveillance objective}, and $\LTLglobally\LTLfinally p_k$, called \emph{liveness surveillance objective}.

\todo{semantics}

\begin{example}
\todo{safety surveillance objective; liveness surveillance objective}
\end{example}


\subsection{Incorporating Functional Objectives}
We can easily integrate functional LTL objectives not related to surveillance by considering in addition to $\SP$ a set $\AP$ of predicates interpreted over states of $G$. In order to define the semantics of $p \in \AP$ over states of $G_\belief$, we require for every $p \in \AP$ that if $(l_a,l_t')$, $(l_a,l_t'')$, and $\vis(l_a,l_t') = \vis(l_a,l_t'') = \false$, then $(l_a,l_t') \models p$ iff $(l_a,l_t'') \models p$. This restriction is without loss of generality, since or a finite set of predicates one can extend the state-space of $G$ by making the predicate valuations part of $L_a$.

\begin{example}
\todo{safety surveillance objective + liveness objective, liveness surveillance objective + liveness objective}
\end{example}


%%%%%%%%%%%%%%%%%%%%%%%%%%%%%%%%%%%%%%%%%%%%%%%%%%%%%%%%%%%%%%%%%%%%%%%%%%%%%%%%

%%%%%%%%%%%%%%%%%%%%%%%%%%%%%%%%%%%%%%%%%%%%%%%%%%%%%%%%%%%%%%%%%%%%%%%%%%%%%%%%


%%%%%%%%%%%%%%%%%%%%%%%%%%%%%%%%%%%%%%%%%%%%%%%%%%%%%%%%%%%%%%%%%%%%%%%%%%%%%%%%
\section{DISTRIBUTED SURVEILLANCE GAMES}
In the sequel we assume that $L_1 = L_2 = \dots = L_t \triangleq L$  in the surveillance game structure, i.e, all $n$ sensors and the target operate in the same state space. For the remainder of the paper, let $G=(\states,s^\init,\trans,\vis_1,\ldots,\vis_n)$ be a multi-agent surveillance game structure  defined over  $L$.

\begin{figure}
A  \emph{state-space partition} of size $n$ of the set $L$ of locations in a game structure $G$ is tuple $\widetilde L =  (\widetilde L_1,\ldots,\widetilde L_n)$ of subsets of $L$ such that  $\bigcup_i^n L_i = L $, and $\widetilde L_i \cap \widetilde L_j  = \emptyset$ for $i \neq j$. 

\subfloat[Surveillance game partitioned into two subgames \label{part-grid}]{
\input{figs/distgames_fig.tex}
\hspace{.3cm}}
%\hfill
\subfloat[Possible transitions from current state in each subgame \label{part-grid-trans}]{

\begin{minipage}{5.0cm}
\vspace{-2cm}
%{\fontsize{8}{10}\selectfont $\Vis((20,4),18) = \false,\vis(4,17) = \false,$ $\vis(4,19) = \true, \vis(4,23) = \false$}

%\smallskip

%\small \[\tilde{L} =\begin{cases}
%\widetilde{L}_1 = \{0,1,2,3,4,5,6,7,8,9\} \\
%\widetilde{L}_2 = \{10,11,12,13,14,\ldots,24\}
%
%\end{cases}\]


\begin{tikzpicture}[node distance=.9 cm,auto,>=latex',line join=bevel,transform shape,scale=.75]
\node at (0,0) (s0) {$G^1: (20,14)$};
\node  [below left of=s0,yshift=-.5cm,xshift=-0.5cm] (s2) {$(21,k_1)$};
\node  [below right of=s0,yshift=-.5cm,xshift=0.5cm] (s3) {$(15,k_1)$};
\node  [left of=s2,xshift=-0.75cm] (s1) {$(21,\{19\})$};
\node  [right of=s3,xshift=0.75cm] (s4) {$(15,\{19\})$};
\draw [->] (s0) edge (s1.north);
\draw [->] (s0) edge (s2.north);
\draw [->] (s0) edge (s3.north);
\draw [->] (s0) edge (s4.north);
\end{tikzpicture}


\begin{center}
\begin{tikzpicture}[node distance=.9 cm,auto,>=latex',line join=bevel,transform shape,scale=.75]
\node at (0,0) (s0) {$G^2: (4,k_2)$};
\node  [below of=s0,yshift=-.5cm] (s1) {$(3,9)$};
\node  [right of=s1,xshift=0.5cm] (s2) {$(9,k_2)$};
\node  [left of=s1,xshift=-0.5cm] (s3) {$(3,k_2)$};
\node  [left of=s3,xshift=-0.5cm] (s4) {$(3,5)$};
\node  [right of=s2,xshift=0.5cm] (s5) {$(9,5)$};
\draw [->] (s0) edge (s1.north);
\draw [->] (s0) edge (s2.north);
\draw [->] (s0) edge (s3.north);
\draw [->] (s0) edge (s4.north);
\draw [->] (s0) edge (s5.north);
\end{tikzpicture}
\end{center}



\end{minipage}
}


\caption{Partitioning of the state space of a surveillance game into two subgames comprising of states $\widetilde{L}_1$ - green and $\widetilde{L}_2 - blue$. }
\label{fig:simple-dist-game}
\end{figure}

\subsection{Surveillance subgames}
We now describe how, given a state-space partition $\widetilde L =  (\widetilde L_1,\ldots,\widetilde L_n)$, to construct a tuple of single-agent surveillance game structures $\widetilde G = (G_1,\ldots,G_n)$ that contains one  \emph{surveillance subgame} $G^i$ for each mobile sensor  $i$. Each subgame, $G^i$ is defined over the subset of locations $\widetilde{L}_i$. Since the target and sensor operate on the same state space we will have $\widetilde{L}^i_s = \widetilde{L}^i_t = \widetilde{L}_i$. Additionally, to each $\widetilde{L}^i_t$ we add an \emph{auxiliary location} $k_i$ that encapsulates all possible locations of the target that are outside of this subgame's region, i.e., all locations in $L \setminus \widetilde{L}_i$.  We then model transitions leaving or entering $\widetilde{L}^i_t$ as transitions to or from location $k_i$ respectively.

More formally, given the subset  $\widetilde{L}_i \subseteq L$ we define the \emph{subgame} of $G$ corresponding to sensor $i$ as the tuple $G^i = (\widetilde{S}_i,\widetilde{s}_i^{init},\widetilde{T}_i,\widetilde{\vis}_i)$  where
\begin{itemize}
\item $\widetilde{S}_i= \widetilde{L}_i \times (\widetilde{L}^i_t \cup k_i)$ is the set of states.
\item The set $\widetilde{T}_i$ consists of two types of transitions: the transitions in $T{\downarrow } i$ that originate and end in locations that are part of the subgame's region are preserved as they are, while the transitions of the target exiting or entering $\widetilde{L}^i_t$ are replaced by transitions to and from location $k_i$ respectively, since $k_i$ represents all target locations outside of  $\widetilde{L}^i_t$. 
Formally, for every pair of states $(\widetilde{l}_i,\widetilde{l}_t) \in \widetilde{S}_i$ and $(\widetilde{l}_i',\widetilde{l}_t') \in \widetilde{S}_i$ we have that $((\widetilde{l}_i,\widetilde l_t),(\widetilde{l}_i',\widetilde l_t')) \in \widetilde T_i$ if and only if there exists a transition
 $((\widetilde{l}_i,l_t),(\widetilde{l}_i',l_t')) \in T{\downarrow}i$ for which the following conditions are satisfied:
 \begin{itemize}
 \item if $\widetilde l_t \in \widetilde L_t ^i$ and $\widetilde l_t' \in \widetilde L_t ^i$, then 
 $\widetilde l_t = l_t$ and $\widetilde l_t'= l_t'$, that is, we have a \emph{transition internal for the region $\widetilde L_t^i$};
 \item if $\widetilde l_t \in \widetilde L_t ^i$ and $\widetilde l_t' =  k_i$, then 
 $l_t \in \widetilde L_t^i$ and $l_t' \not\in \widetilde L_t^i$, that is, we have a \emph{transition exiting the region $\widetilde L_t^i$}; 
 \item if $\widetilde l_t= k_i$ and $\widetilde l_t' \in  \widetilde L_t ^i$, then 
 $l_t \not \in \widetilde L_t^i$ and $l_t' \in \widetilde L_t^i$, that is, we have a \emph{transition entering the region $\widetilde L_t^i$}; 
 \item if $\widetilde l_t= k_i$ and $\widetilde l_t' =  k_i$, then 
 $l_t \not \in \widetilde L_t^i$ and $l_t' \not\in \widetilde L_t^i$, that is, we have a \emph{transition completely outside $\widetilde L_t^i$}.
\end{itemize}  

  \item The visibility function $\widetilde{\vis}_i$ in the subgame $G^i$ agrees with the visibiity function $\vis_i$ of sensor $i$ in the original game when the target's location is in the subgame's region. Target locations outside of the region $\widetilde L_t^i$  (summarized by location $k_i$) are invisible to the sensor in the subgame. Formally, 
 \[\widetilde{\vis}_i(\widetilde{l}_i,l_t) = \begin{cases}
\vis_i(\widetilde{l}_i,l_t) & \text{if } l_t \in \widetilde{L}_t^i, \\
\false & \text{if } l_t  = k_i.
\end{cases}
\]
\end{itemize}
\begin{example}
In Figure \ref{part-grid}, we have two subgames: $G^1$ for the green sensor and $G^2$ for the blue one. The states in the subgames are $s_1 = (20,14)$, and $s_2 = (4,k_2)$ . Recall that $k_i$ is an indicator state to represent that the target is not subgame $i$. The possible transitions shown in \ref{part-grid-trans} show that the target has the ability to either leave $G^1$ and enter $G^2$. 
%Finally recall that in example \ref{ex:simple-surveillance-game}, we had state $Vis((20,4),14) = \true$. This is no longer the case as although we have $\vis_1(4,14) = \true$ in the global game (i.e, it is visible to sensor 1 and hence also visible to the entire sensor network), state 14 does not lie in $\widetilde{L}_1$ which means it is not visible in the subgame for sensor 1.
\qed
\end{example}

Note that in this construction, sensor $i$ is not able to leave the region of locations $\widetilde{L}_i$. Furthermore, all the information about the target's behaviour outside of  the subgame's region is completely hidden from the mobile sensor controller, since all locations outside of  $\widetilde{L}_t^i$ are represented by the single location $k_i$.
%Finally we remark that the previously defined notions of $\succs_t(\widetilde{l}_i,\widetilde{l}^i_{t})$, $\succs_t(\widetilde{l}_i,\widetilde{L}^i_{t})$, and $\succs(\widetilde{l}_i,\widetilde{l}^{t},\widetilde{l}_{t}')$ for the global game structure $G$ follow analogously in this construction for $\widetilde{T}$ which we denote as $\widetilde{\succs}$.
In the next subsection we discuss the local knowledge (belief)  of sensor $i$ in the game structure $G^i$.

\subsection{Local beliefs in surveillance subgames}\label{sec:local-games}
A surveillance subgame is a game structure with a single mobile sensor, and thus, can be seen as a special case of multi-agent surveillance game structure. With this, the definition of  belief-set game structures from Section~\ref{sec:belief-gs} directly applies to surveillance subgames.

In the belief-set game structure for a surveillance subgame $G^{i} = (\widetilde{S}_i,\widetilde{s}_i^{init},\widetilde{T}_i,\widetilde{\vis}_i)$ the belief sets represent the local belief of sensor $i$. More specifically, a belief set in $G^{i}_\belief$ is an element of  $\mathcal{P}(\widetilde{L}^i_t \cup \{k_i\})$, and can thus contain the auxiliary location. Intuitively, if $ k_i$ is present in the sensor's current belief, then the target could possibly be outside of the local set of locations $\widetilde L_i$ , or if the belief is the singleton $\{k_i\}$, then sensor $i$ knows for sure that the target is outside of its region.  
We define the \emph{global interpretation} $\llbracket B_t\rrbracket$ of a belief set $B_t$ in $G^{i}_\belief$, which is a set of locations in $G$, as
 \[\llbracket B_t\rrbracket = \begin{cases}
B_t & \text{if } k_i \not\in B_t, \\
B_t \cup (L \setminus \widetilde L_i) & \text{if } k_i  \in B_t.
\end{cases}
\]

%belief subgame  $G^{i}_\belief= (\widetilde{\states}_{i_\belief},\widetilde{s}^\init_{i_\belief},\widetilde{\trans}_{i_\belief})$ 
%\begin{itemize}
%\item $\states_{i_\belief} = \widetilde{L}_i \times \mathcal{P}(\widetilde{L}^i_t \cup \{k_i\})$ is the set of states,
%\item $\widetilde{\trans}_{i_\belief} \subseteq \widetilde{\states}_{i_\belief} \times \widetilde{\states}_{i_\belief}$ is the transition relation where $((\widetilde{l}_i, B_t),(\widetilde{l}_i', B_t')) \in \widetilde{\trans}_{i_\belief}$ iff $\widetilde{l}_i' \in  \widetilde{\succs}(\widetilde{l}_i,l_t,l_t')$ for some $l_t \in B_t$ and $l_t' \in B_t'$ and one of these holds:
%\begin{itemize}
%\item[(1)] $B_t' = \{l_t'\}$, $l_t' \in \widetilde{\post}(\widetilde{l}_i,B_t)$, $\widetilde{\vis}_i(l,l_t') = \true$;
%\item[(2)] $B_t' = \{l_t' \in \widetilde{\post}(\widetilde{l}_i,B_t)  \mid  \widetilde{\vis}_i(l,l_t') = \false \}$.
%\end{itemize}
%\end{itemize}
% For brevity, we omit the definitions of $\widetilde{\succs}_t$, $\widetilde{\succs}$, runs and strategies for the belief subgames as the definitions follow trivially from those for the belief games but in the subspace of transitions and states as defined above.

Strategies of sensor $i$ in the belief-set game $G^{i}_\belief$ depend only on the sequence of states in this game, and thus, only on local information. Following the definitions in Section~\ref{sec:belief-gs}, the outcome of a pair of given strategies $f_{i}$ and $f_{t_i}$ for the sensor and the target in $G^{i}_{\belief}$  is a sequence of  states in  $G^{i}_{\belief}$, each of which is a pair consisting of a location of sensor $i$ and a belief-set for sensor $i$ in $G^{i}_{\belief}$.

%\paragraph*{Remark}$B^k_{t_i}$ is the belief that sensor $i$ holds on the position of the target. Since the subgame for each sensor will be executing concurrently, the global belief on the location of the target will be a combination of the knowledge of the individual sensors. This is discussed in more detail in the sequel.

\subsection{Composition of subgames}
Given $n$ belief subgames $(G^{1}_{\tilde{S}_\belief},\ldots,G^{n}_{\tilde{S}_\belief})$ over state partition $\widetilde{L}$, we define a \emph{run} in the composed belief game $G_\belief$ as an infinite sequence $s_0,s_1,\ldots$ of states where $s_k = (\tilde{s}_{1_k} \ldots \tilde{s}_{n_k})$ is the combined state of all the subgames at timestep $k$, and  $(\tilde{s}_{i_k},\tilde{s}_{{i+1}_k}) \in \tilde{T}_{i_\belief}$ for all $k$. 

Given a strategy for each sensor $f_{s_i}$ in the corresponding subgames, we compose the subgame strategies into a global strategy in the same way. 


\subsection{Global belief vs local belief}

Given $n$ belief subgames, at timestep $k$, sensor $i$ holds belief $B^k_{t_i}$. This is the \emph{local} belief as it is held by a single sensor and is not shared with the others. We first present the following fact: \todo{\textbf{not sure if this is a theorem, but not sure what else to call it.}}
\begin{theorem} 
At timestep $k$, if there exists $i\in \{1\dots n\}$ such that $B^k_{t_i} = \{l_t\}$ for $l_t \in \widetilde{L}_i^t$, then for all $j \neq i$, we have $k_j \in B^k_{t_j}$.
\end{theorem}
\begin{proof}
By construction of the subgames \textbf{more to come}
\end{proof}
\begin{corollary}\label{corr:uniqi}
There can exist at most one $i\in \{1\dots n\}$ such that $B|^k_{t_i}| = 1$ and $k_j \in B^k_{t_j} \notin B|^k_{t_i}|$.
\end{corollary}
Intuitively, this states that if the target is in vision of sensor $i$, all the other sensors must have $k_j$ in their belief sets. Recall that $k_j$ indicates that the target is not in the state space of the subgame corresponding to sensor $j$. The corollary states that if the target is in vision of sensor $i$, then no other sensor can claim to see the same target. 
The global belief is defined as:
\[B^k_t \triangleq 
\bigcap_{i}^n B^k_{t_i}
\]

 




\subsection{Distributed surveillance synthesis problem}
\paragraph*{\textbf{Problem statement: }}Given a global surveillance game $(G,\varphi)$ with $n$ sensors and a state partition $\widetilde{L}$, compute a distributed strategy for the sensor in each subgame $G^i_\belief$ such that the composition of the $n$ subgame strategies guarantees that the global surveillance objective $\varphi$ is satisfied. Formally, compute the strategy $f_i$ for the sensor in each belief subgame such that $f_s = f_{s_1} \bigoplus f_{s_2} \dots \bigoplus f_{s_n}$ must be winning for the global property, i.e, $\outcome(G_{\belief},f_{s},f_{t}) \models \varphi$

%if $\outcome(G^i_\belief},f_{s_i},f_{t_i}) \models \varphi_i$, then the composed strategy for all agents 
%\Rayna{The definition above mixes up several things (1) the distributed surveillance synthesis problem where each sensor has only a local objective (2) the reduction from a given multi-agent surveillance synthesis problem (satisfying some restriction) to a distributed surveillance synthesis problem with local objectives such that  the composition of the resulting strategies satisfies the global objectives. This property of the reduction is something we have to prove.}






\section{FROM GLOBAL TO LOCAL SPECIFICATIONS}
%\subsection{From global to local surveillance specifications}
\everypar{\looseness=-1}
In order to reduce the multi-agent surveillance synthesis problem for a given surveillance specification $\varphi$ to solving a number of single-sensor surveillance subgames, we need to provide local surveillance objectives for the individual subgames. The local objectives should be such that by composing the strategies that are winning with respect to the local objectives we should obtain a strategy that is winning for the global surveillance objective. More precisely, we have to provide local surveillance specifications $\varphi_1,\ldots,\varphi_n$ such that if for each $i$ it holds that $f_{s_i}$ is  a winning strategy for the sensor in $(G^i,Q_i,\varphi_i)$, then the strategy $f_{s_1}\otimes \ldots \otimes f_{s_n}$ is  a joint winning strategy for the sensors in $(G,\mathcal M,\varphi)$.

Recall that the surveillance objective $\varphi$ is of the form $\LTLglobally p_b$, or $\LTLglobally\LTLfinally p_b$, or $\LTLglobally p_a \wedge \LTLglobally\LTLfinally p_b$, where $a > b$. We will provide translations for each of these types of  specifications.

First, note that the belief sets in a belief subgame $G^i_\belief$ can contain the auxiliary location $k_i$, which represents all locations in $L \setminus \widetilde L_i$. Thus, when the local belief set contains $k_i$, the size of the global belief set depends on the local beliefs of the other agents as well. We have to account for this in the translation from global into local surveillance objectives.

\begin{example}\label{ex:global-local-safety}
Consider the global safety surveillance specification $\LTLglobally p_5$ in a network with two mobile sensors. In this case we can reduce the multi-agent surveillance problem to two single-agent surveillance games, each of which has $\LTLglobally p_3$ as the local specification. 
\everypar{\looseness=-1}
To see why, consider the two possible cases of local belief set of sensor $1$ whose size is less than or equal to $3$. If $k_1$ is not part of the belief set of sensor $1$, then the target is definitely in the region of sensor $1$, meaning that the global belief is of size less than or equal to $3$, and hence smaller that $5$. If, on the other hand, $k_1$ is part of the local belief of sensor $1$, then the target can be in at most $2$ locations in $\widetilde L_1$. If at the same time we have that the local belief of sensor $2$ is of size at most $3$, this would guarantee that the size of the global belief does not exceed $5$. 
\everypar{\looseness=-1}
Local specifications $\LTLglobally p_4$, on the other hand do not imply the global specification. Indeed, if at a given point in time both sensors have local beliefs of size $4$, each of which contains the corresponding location $k_i$, the resulting global belief will be of size $6$ and thus violate the global specification.\qed
\end{example}

Generalizing the observations made in this example, for any number of sensors $n \geq 2$ and global safety surveillance objective $\LTLglobally p_b$, we define the local safety surveillance objective for each of the sensors, denoted $\locspec(\LTLglobally p_b,n)$, as $\locspec(\LTLglobally p_b,n) \triangleq \LTLglobally p_c,$ where $c = \lfloor{\frac{b}{n}}\rfloor+1$. Since $n \geq 2$ and $b >0$, we have $c \leq b$.

Note that this translation is conservative, since if according to the belief of sensor $i$ the target could be outside its region, it should guarantee that the number of locations in its own region the target could be in is at most  $\lfloor{\frac{b}{n}}\rfloor$, even if the target can possibly be in only one of the other regions. This conservativeness is necessary to guarantee soundness in the absence of communication between the sensors.

We now turn to liveness surveillance objectives. It is easy to see that each sensor guaranteeing a small enough local belief infinitely often is not enough to satisfy the global surveillance objective, since the local guarantees can happen in time-steps different for the different sensors.

\begin{example}\label{ex:global-local-liveness}
Consider the global surveillance specification $\LTLglobally\LTLfinally p_5$ for a network with two sensors. Suppose $f_1$ is a strategy for the sensor in $G^1_\belief$,  which ensures that every even step the size of the local belief is $10$, and every odd step the local belief contains $k_1$ and its size is $3$. Strategy  $f_2$ in $G^2_\belief$,  is similar, but even and odd steps are interchanged: every even step the local belief contains $k_2$ and its size is $3$, and every odd step the size of the local belief is $10$.  Thus, while $f_1$ and $f_2$ guarantee that their local belief is "small enough" infinitely often, they do this at different steps.
\end{example}

We circumvent the problem illustrated in this example by requiring that each sensor satisfies the liveness guarantee on its own. For this, we have to consider two cases. First, if from some point on sensor $i$ always knows that the target is outside of its region, it has no obligation to satisfy the liveness surveillance guarantee. If, on the other hand, according to sensor $i$'s belief the target could be in $\widetilde L_i$ infinitely often (note that this is true for at least one sensor), then it has to satisfy the corresponding liveness guarantee.

In order to capture this intuition, we need two additional types of surveillance predicates. First, we need to be able to express the negation of the property that the local belief of sensor $i$ is the singleton $\{k_i\}$ (which means that sensor $i$ knows that the target is outside $\widetilde L_i$). For this, we introduce the predicate $\mathit{belief} \neq \{k_i\}$. Second, in order to express the local liveness guarantee, we need to be able to state that $k_i$ is not in $\widetilde L_i$ (which means that sensor $i$ knows that the target is in its region). The predicate we introduce for this property is $k_i \not\in\mathit{belief}$. Both predicates can be interpreted over belief sets similarly to $p_b$ and incorporated in LTL.
 
Formally, we define the local liveness specification for sensor $i$ denoted $\locspec_i(\LTLglobally\LTLfinally p_b)$ as
\[
\begin{array}{lll}
\locspec_i(\LTLglobally\LTLfinally p_b) & \triangleq &
\big(\LTLglobally\LTLfinally (\mathit{belief} \neq \{k_i\} )\big)\rightarrow\\&& \big(\LTLglobally\LTLfinally (p_b \wedge (k_i \not\in\mathit{belief}))\big).
\end{array}
\]
%Notice that here we use implication, which is not part of the fragment to which we restrict the global surveillance specifications. This is not a problem, since our surveillance strategy synthesis procedure for single-sensor games is applicable to such surveillance formulas as well.
Note that the agent cannot trivially satisfy $\locspec_i(\LTLglobally\LTLfinally p_b)$, since the belief set is defined precisely by it's sequence of observations and is not under the agent's direct control.

This translation is again conservative, since it would suffice that the liveness guarantee is satisfied by a single sensor. However, these can be different sensors for different behaviours of the target. Thus, we require that \emph{every} sensor $i$ satisfies $\locspec_i(\LTLglobally\LTLfinally p_b)$. This requires that if the target crosses from one region to another infinitely often, then both sensors have to satisfy the liveness surveillance objective.

Finally, for a global surveillance specification $\LTLglobally p_a \wedge \LTLglobally\LTLfinally p_b$, the local surveillance specification for sensor $i$ is
\[\locspec_i(\LTLglobally p_a\wedge  \LTLglobally\LTLfinally p_b,n)\triangleq
\locspec(\LTLglobally p_a,n) \wedge\locspec_i(\LTLglobally\LTLfinally p_b) .
\]

Slightly abusing the notation, we denote with $\locspec_i(\varphi,n)$ the local surveillance specification for sensor $i$ for any of the three types of global surveillance specifications.

The next theorem, which follows  from the definition of the local specifications, states the soundness of the reduction.

\begin{theorem}\label{thm:soundnes}
Let  $(G,\mathcal M,\varphi)$ be a multi-agent surveillance game with $n$ sensors, where $\varphi$ is of the form $\LTLglobally p_b$, or $\LTLglobally\LTLfinally p_b$, or $\LTLglobally p_a \wedge \LTLglobally\LTLfinally p_b$, where $a > b$. Let $\widetilde{L}$ be a state-space partition. Suppose that $f_1,\ldots,f_n$ are strategies for the sensors in the subgames $G^1_\belief,\ldots,G^n_\belief$ respectively, such that for each sensor $i$ the strategy $f_i$ is winning in the surveillance game $(G^i,Q_i,\locspec_i(\varphi,n))$. Then, it holds that the composed strategy $f_{s_1}\otimes\ldots\otimes f_{s_n}$ is a joint winning strategy for the sensors in the surveillance game $(G,\mathcal M,\varphi)$.
\end{theorem}
%
%\section{DECENTRALIZED SYNTHESIS}
%We are now ready to solve each game $G^i$ with local surveillance objective $\locspec_i$. However, for a controller to 

\section{EXPERIMENTAL EVALUATION}\label{sec:experiments}
We now return to the case study outlined in Section~\ref{sec:casestudy}. We have implemented the proposed method in \texttt{Python}, using the \texttt{slugs} reactive synthesis tool~\cite{EhlersR16}, and evaluated it on the multi-agent surveillance game modelling the problem described in Section~\ref{sec:casestudy}. The experiments were performed on an Intel i5-5300U 2.30 GHz CPU with 8 GB of RAM.

 We analyzed two scenarios. In Figure \ref{fig:experiment}, we have six \emph{mobile sensors}. We compare the surveillance strategy with the situation in Figure \ref{fig:3experiment} where we have three \emph{mobile sensors}. In both cases there are four \emph{static sensors} depicted in yellow in Figure~\ref{fig:bigexp}. Our global surveillance task is $\LTLsquare \LTLdiamond p_5$, i.e, we need to infinitely often bring the belief of the target location to 5 cells or lower. 

\begin{figure}
	\centering
\subfloat[The gridworld in \ref{fig:SGR-grid}\newline partitioned into 6 subgames. \label{fig:experiment}]{
\includegraphics[scale=0.18]{figs/SGR-grid-vis-part.png}
\hspace{1.5cm}}
%\hfill
\subfloat[The gridworld in \ref{fig:SGR-grid}\newline partitioned into 3 subgames. \label{fig:3experiment}]{
\includegraphics[scale=0.18]{figs/SGR-grid-vis-part_3.png}
\hspace{.3cm}}

\caption{Cases with 6 mobile sensors in Fig \ref{fig:experiment} and 3 mobile sensors in Fig \ref{fig:3experiment}. The mobile sensors are blue circles and the target is represented in orange. Yellow regions represent static sensors.The red cells represent impassable terrain (such as dense foliage) that cannot be seen through by the sensors. Black cells are locations not visible to any sensor.}\label{fig:bigexp}\vspace{-0.5cm}
\end{figure} 

Solving either case centralized is not computationally feasible as the state space grows exponentially with the number of sensors - we will have in the order of $400^6$ and $400^3$ states respectively. Thus, we partition the multi-agent surveillance game into subgames as shown in Figures \ref{fig:experiment} and \ref{fig:3experiment}. We then solve each game individually with local specifications $\locspec_i(\LTLglobally\LTLfinally p_5)$. We solve these \emph{single-agent} surveillance games using an abstraction-based method detailed in a companion publication at CDC 2018, detailed in~\cite{arxiv}. 
We report the synthesis times in Table \ref{tab:synthtime}.

\begin{table}[h!]
\vspace{-0.2cm}
	\centering
	\caption{Synthesis times for each surveillance subgame}
	\label{tab:synthtime}
	\begin{tabular}{c|l|l|l}
		\multicolumn{1}{l|}{}                                    & \textbf{Subgame} & \textbf{Number of states} & \textbf{Synthesis time (s)} \\ \hline \hline
		\multirow{7}{*}{\textbf{6 sensors}}
		& Subgame 1   & 69     & 101                          \\
	    & Subgame 2   & 74     & 206                          \\
		& Subgame 3   & 62     & 111                          \\
		& Subgame 4   & 52     & 88                          \\
		& Subgame 5   & 77     & 285                          \\
		& Subgame 6   & 66     & 64                          \\ \hline
		& \textbf{Total}   & \textbf{400}         & \textbf{855}                         \\ \hline
		\multicolumn{1}{l|}{\multirow{4}{*}{\textbf{3 sensors}}} & Subgame 1        & 142 & 473                         \\
		\multicolumn{1}{l|}{}                                    & Subgame 2        & 113 & 306                         \\
		\multicolumn{1}{l|}{}                                    & Subgame 3        & 145 & 372                         \\ \hline
		\multicolumn{1}{l|}{}                                    &  \textbf{Total} & \textbf{400}            & \textbf{1151}                        
	\end{tabular}
\end{table}
\vspace{-0.2cm}
The multi-agent surveillance game in Figure \ref{fig:experiment} results in more subgames compared to the game in \ref{fig:3experiment}. However, each game is much smaller and strategies can be synthesized faster in each subgame. Figure \ref{fig:case1exp} shows snapshots in time of the simulation of the 3 sensor surveillance game in Figure \ref{fig:3experiment}. The target is being controlled by a human and the sensors are following their synthesized local surveillance strategies. The global belief is depicted in Figure \ref{fig:case1exp} as grey cells, meaning that the combined knowledge of all the sensors has restricted the location of the target into one of the grey cells.
\begin{figure}
	
	\begin{minipage}{5.0cm}
		\centering
		\subfloat[$t_8$ \label{fig:case1t2}]{
			\includegraphics[scale=0.12]{figs/results_t2.png}\hspace{.7cm}
		}
		\subfloat[$t_{12}$ \label{fig:case1t3}]{
			\includegraphics[scale=0.12]{figs/results_t3.png}\hspace{.7cm}
		}
		\subfloat[$t_{16}$ \label{fig:case1t4}]{
			\includegraphics[scale=0.12]{figs/results_t4.png}\hspace{.7cm}
		}
	\end{minipage}
	\begin{minipage}{5.0cm}
		\centering
		\subfloat[$t_{18}$  \label{fig:case1t5}]{
			\includegraphics[scale=0.12]{figs/results_t1.png}\hspace{.7cm}
		}
		\subfloat[$t_{20}$ \label{fig:case1t6}]{
			\includegraphics[scale=0.12]{figs/results_t5.png}\hspace{.7cm}
		}
		\subfloat[$t_{22}$ \label{fig:case1t7}]{
			\includegraphics[scale=0.12]{figs/results_t6.png}\hspace{.7cm}
		}
		
	\end{minipage}
	
	
	\caption{Figures \ref{fig:case1t2} - \ref{fig:case1t7} are chronological snapshots during simulation of the surveillance game in Figure \ref{fig:3experiment}. Grey regions represent the global belief of the target's location.  
	}\vspace{-0.5cm}
	\label{fig:case1exp}
	
\end{figure}

 We see, in Figures \ref{fig:case1t2} - \ref{fig:case1t4}, that the target is in the subgame corresponding to sensor 2. Hence, only sensor 2 is moving and trying to lower its belief to below 5 cells (which it does in Figure \ref{fig:case1t5}). In Figures \ref{fig:case1t3} - \ref{fig:case1t5}, the target starts moving towards subgame 3 at which point the target is detected by the static sensor in subgame 3 and sensor 3 takes over in figures \ref{fig:case1t6} - \ref{fig:case1t7}. There is no communication between any of the agents, and each satisfy only their local surveillance specification. However, our construction guarantees that the global specification of $\LTLfinally \LTLglobally p_5$ will be satisfied. %We include a link to a video of the simulation at \url{google.com}
 
 %In figure \ref{fig:case1t2}, the target is in the region of a static sensor. Hence, the global belief is restricted to the region that sensor operates. 

%%%%%%%%%%%%%%%%%%%%%%%%%%%%%%%%%%%%%%%%%%%%%%%%%%%%%%%%%%%%%%%%%%%%%%%%%%%%%%%%

\section{CONCLUSIONS}
We presented a method for decentralized synthesis of surveillance strategies for a mobile sensor network working with static sensors. Problems that would otherwise be computationally intractable can be solved by decomposing the global game into local subgames for each sensor with individual surveillance specifications. We show that though each game is solved completely independently with no information sharing, we can still guarantee global surveillance properties.

For future work, we aim to incorporate false positives in static alarm triggers as well as noisy observations from the mobile sensors while still guaranteeing surveillance specifications.

%%%%%%%%%%%%%%%%%%%%%%%%%%%%%%%%%%%%%%%%%%%%%%%%%%%%%%%%%%%%%%%%%%%%%%%%%%%%%%%%

%\section*{APPENDIX}

%%%%%%%%%%%%%%%%%%%%%%%%%%%%%%%%%%%%%%%%%%%%%%%%%%%%%%%%%%%%%%%%%%%%%%%%%%%%%%%%

%\section*{ACKNOWLEDGMENT}


%%%%%%%%%%%%%%%%%%%%%%%%%%%%%%%%%%%%%%%%%%%%%%%%%%%%%%%%%%%%%%%%%%%%%%%%%%%%%%%%
\bibliographystyle{IEEEtran}
\bibliography{main}

\end{document}
