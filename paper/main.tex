\documentclass[letterpaper, 10 pt, conference]{ieeeconf}  % Comment this line out if you need a4paper

%\documentclass[a4paper, 10pt, conference]{ieeeconf}      % Use this line for a4 paper

\IEEEoverridecommandlockouts                              % This command is only needed if 
                                                          % you want to use the \thanks command

\overrideIEEEmargins                                      % Needed to meet printer requirements.

%\renewcommand{\baselinestretch}{2}

% See the \addtolength command later in the file to balance the column lengths
% on the last page of the document

\usepackage{amsmath} 
\usepackage{amssymb} 


\usepackage{ltl} 
\usepackage{multirow}
\usepackage[vlined]{algorithm2e}
\usepackage{mathtools}

\usepackage{color}
\usepackage{subfig}
\usepackage{tikz}
\usepackage{hyperref}
\usetikzlibrary{arrows,fit,shapes,automata}
\usetikzlibrary{positioning,fit,calc,shapes}
\usetikzlibrary{decorations.fractals}
\usetikzlibrary{decorations.markings}
\usepackage{stmaryrd}

\newcommand{\Rayna}[1]{{\textcolor{magenta}{ \textbf{Rayna:} #1 $\spadesuit$ }}}
\newcommand{\Suda}[1]{{\textcolor{blue}{ \textbf{Suda:} #1 $\spadesuit$ }}}
\newcommand{\Ufuk}[1]{{\textcolor{red}{ \textbf{Ufuk:} #1 $\spadesuit$ }}}
\newcommand{\todo}[1]{{\textcolor{red}{TODO:} #1}}
\newcommand{\comment}[1]{}

\newtheorem{example}{Example}
\newtheorem{theorem}{Theorem}
\newtheorem{corollary}{Corollary}
\newtheorem{proposition}{Proposition}
\newcommand*{\qed}{\hfill\ensuremath{\blacksquare}}

\newcommand{\init}{\mathsf{init}}
\newcommand{\belief}{\mathsf{belief}}
\newcommand{\abstr}{\mathsf{abstract}}
\newcommand{\vis}{\mathit{vis}}
\newcommand{\Vis}{\mathit{Vis}}
\newcommand{\succs}{\mathit{succ}}
\newcommand{\beliefs}{\mathcal{P}(L_t)}
\newcommand{\subbeliefs}{\mathcal{P}(\tilde{L}^i_t)}
\newcommand{\Surveillance}{\mathsf{Surveillance}}
\newcommand{\beliefF}{\mathit{belief}}

\newcommand{\states}{S}
\newcommand{\trans}{T}
\newcommand{\part}{\mathcal{Q}}

\newcommand{\post}{\mathit{succ}_t}

\newcommand{\outcome}{\mathit{outcome}}
\newcommand{\counterex}{\mathcal{C}}

\newcommand{\bools}{\mathbb{B}}
\newcommand{\true}{\mathit{true}}
\newcommand{\false}{\mathit{false}}
\newcommand{\nats}{\mathbb{N}}

\newcommand{\SP}{\mathcal{SP}}
\newcommand{\AP}{\mathcal{AP}}

\newcommand{\locspec}{\mathit{local}}





   

\title{\LARGE \bf Synthesis of Surveillance Strategies via Belief Abstraction}

\author{Suda Bharadwaj$^{1}$ and Rayna Dimitrova$^{1}$ and Ufuk Topcu$^{1}$% <-this % stops a space
%\thanks{*This work was not supported by any organization}% <-this % stops a space
\thanks{$^{1}$Suda Bharadwaj, Rayna Dimitrova and Ufuk Topcu are with the University of Texas at Austin}%
%\thanks{$^{2}$Bernard D. Researcheris with the Department of Electrical Engineering, Wright State University, Dayton, OH 45435, USA {\tt\small b.d.researcher@ieee.org}}%
}



\begin{document}

\maketitle
\thispagestyle{empty}
\pagestyle{empty}

%%%%%%%%%%%%%%%%%%%%%%%%%%%%%%%%%%%%%%%%%%%%%%%%%%%%%%%%%%%%%%%%%%%%%%%%%%%%%%%%

\begin{abstract}
We study the problem of synthesizing a controller for a robot given a linear temporal logic specification, as well as a surveillance objective where it is required to maintain knowledge of the location of a moving, possibly adversarial target. We formulate this problem as a one-sided partial-information game. We then reduce the partial-information game to a full-observation game using an abstracted belief set construction to avoid the exponential blow up of the state space. In order to ensure the surveillance requirement is satisfied, we allow it to be encoded in three different ways : a safety objective, a liveness objective, or both depending on the qualitative behavioural requirement for the user. This requirement is then appended to the original specification and formulated as an assume-guarantee problem for reactive synthesis on the abstract belief-set game. We then use a counterexample-guided abstraction refinement scheme to refine the abstract belief states until a reactive controller can be synthesized.
\end{abstract}

%%%%%%%%%%%%%%%%%%%%%%%%%%%%%%%%%%%%%%%%%%%%%%%%%%%%%%%%%%%%%%%%%%%%%%%%%%%%%%%%

\section{INTRODUCTION}
Performing surveillance on an adversarial target, by its very nature is a partial information problem. The agent may not always know the position of the target. However, surveillance in conjunction with a mission specification can be crucial in applications such as defense where it is important to keep track of (potentially hostile) targets whilst trying to satisfy a particular objective. 

Since we are dealing with an adversarial target a natural setting for formulating the problem is a two-player game. There are several flavours of partial information games that have been studied in the literature \cite{Chatterjee2013}, and in this paper we focus on turn-based one-sided partial-observation deterministic games on which we perform reactive control synthesis. It is one-sided as we allow the adversary full information on the location of the agent even if it is not in sight. 

Our aim is to then synthesize a reactive controller that satisfies both the LTL specification as well as the surveillance objective. While it has been shown that for a general LTL specification, the synthesis problem is doubly exponential in the length of the formula \cite{Pnueli1989}, the work in \cite{Piterman2006} lays out a class of formulae called GR(1) that is $\mathcal{O}(N^3)$. This framework has been used extensively in robotic planning, for example in \cite{wong2012,Kress2007} and we will use the same here. We explicitly encode the surveillance requirement into the GR(1) formula to allow for synthesis. 

However, we still have the issue of partial observability in our setting. The controller will need to choose actions even when the state of the adversary is not known. The standard approach to deal with the partial observability is by using a \emph{belief set construction} to reduce the problem to a full observability game \cite{Bertoli2006}. However, the number of belief states will be exponential in the number of states \cite{Rintanen2004} as the belief set construction takes a powerset of the number of states. In general, this will scale badly and not be usable for most practical situations. \todo{literature on other partial information reduction heuristics}

In this paper, to deal with this problem, we introduce \emph{abstract belief set construction}. This is an underapproximation of the true belief space and hence, if a controller is found, then we know a controller will exist in the fully refined belief space. If a controller is not found, we use counterexample guided abstract refinement (CEGAR) to split a belief set and the process is repeated. While CEGAR is used on abstract models for GR(1) reactive synthesis, to our knowledge it has not been used on belief state refinement in converting a partial information game to a full information game. 



Our contributions in this paper are as follows:
\begin{itemize}
\item We encode the surveillance task as a safety specification which forces the agent to more closely follow the adversary in order to ensure the uncertainty (size of the belief set) on the location of adversary does not grow above the constraint.
\item We also encode surveillance task as a liveness objective. This allows for the agent to be more relaxed in monitoring the location of the agent if it can ensure that it can see it again sometime in the future.
\item We analyse the qualitatively different behaviour produced based on the specification type which allows the user to tailor the specification based on the requirements of the mission.
\item Avoiding the state space blow up by abstract belief set construction and using counter example guided belief refinement for both the safety and liveness specification cases.
\end{itemize}




%%%%%%%%%%%%%%%%%%%%%%%%%%%%%%%%%%%%%%%%%%%%%%%%%%%%%%%%%%%%%%%%%%%%%%%%%%%%%%%%

\section{GAMES WITH SURVEILLANCE OBJECTIVES}
\subsection{Surveillance Game Structures}
We define a \emph{surveillance game structure} to be  a tuple $G  = (\states,s^\init,T,\vis)$, where:
\begin{itemize}
\item $\states = L_a \times L_t$ is the set of states with $L_a$ the set of locations of the agent and $L_t$ the locations of the target;
\item $s^\init = (l_a^\init,l_t^\init)$ is the initial state;
\item $\trans \subseteq \states \times \states$ is the transition relation describing the possible moves of the agent and the target;
\item $\vis : \states \to \bools$ is a function that maps a state $(l_a,l_t)$ to $\true$ if and only if\emph{ $l_t$ is in the line of sight of $l_a$}.
\end{itemize}

\begin{example}
\todo{grid plus small part of corresponding game structure}
\end{example}
\subsection{Belief-Set Game Structures}

For a surveillance game structure $G  = (\states,s^\init,T,\vis)$ we define the corresponding \emph{belief game structure} $G_\belief  = (\states_\belief,s^\init_\belief,T_\belief)$ with the following components
\begin{itemize}
\item $\states_\belief = L_a \times \mathcal P(L_t)$ is the set of states with $L_a$ the set of locations of the agent, and $\mathcal P(L_t)$ the set of belief sets describing information about the location of the target;
\item $s^\init_\belief = (l_a^\init,\{l_t^\init\})$ is the initial state;
\item $\trans_\belief \subseteq \states_\belief \times \states_\belief$ is the transition relation such that $((l_a, B_t),(l_a', B_t')) \in T_\belief$ if and only if one of the following two conditions is satisfied:
\begin{itemize}
\item $B_t' = \{l_t'\}$ for some $l_t'$ such that $\vis(l_a,l_t') = \true$ and
there exists $l_t \in B_t$ with $((l_a,l_t),(l_a',l_t')) \in T$;
\item $\begin{array}{lll}
B_t' = \{l_t' & \mid & \vis(l_a,l_t') = \false \text{ and }\\
&&\exists l_t \in B_t.\; ((l_a,l_t),(l_a',l_t')) \in T\}.
\end{array}
$
\end{itemize}
The first condition captures the successor locations of the target that can be observed from the agent's current location, while the second corresponds to the belief set consisting of all possible successor locations of the target not visible from the agent's current location.
\end{itemize}

\noindent{\bf Remark} \todo{explain sequence of agent's observation and action}
\begin{example}
\todo{small part of corresponding belief-set game structure}
\end{example}


With each state $(l_a,B)$ in $\states_\belief$ we associate a set of states in $G$ defined as $\gamma((l_a,B)) = \{(l_a,l_t) \mid l_t \in B\}$.

\subsection{Temporal Quantitative Surveillance Objectives}

We consider a set of \emph{surveillance predicates} $\SP = \{p_k \mid k \in \nats\}$, where for $k \in \nats$ we say that a state $(l_a,B_t)$ in the belief game structure satisfies $p_k$ (denoted $(l_a,B_t) \models p_k$) iff 
$|\{l_t \in B_t \mid \vis(l_a,l_t)  = \false \}| \leq k$. Intuitively, $p_k$ is satisfied by the states in the belief game structure where the size of the belief set does not exceed the threshold $k \in \nats$.

We consider surveillance objectives expressed by formulas of linear temporal logic (LTL) over surveillance predicates. LTL surveillance formulas are generated by the grammar
\[\varphi := p \;\mid\; \neg \varphi \;\mid\; \varphi \vee \varphi \;\mid\; \LTLnext  \varphi  \;\mid\; \varphi \LTLuntil \varphi,\]

where $p \in \SP$ is a surveillance predicate, $\LTLnext$ is the \emph{next} operator, and $\LTLuntil$ is the \emph{until} operator. We also define the derived operators 
\emph{finally} $\LTLfinally \varphi = \true \LTLuntil \varphi$ and 
\emph{globally} $\LTLglobally \varphi = \neg \LTLfinally \neg \varphi$.
Of special interest will be formulas of the form $\LTLglobally p_k$, termed \emph{safety surveillance objective}, and $\LTLglobally\LTLfinally p_k$, called \emph{liveness surveillance objective}.

\todo{semantics}

\begin{example}
\todo{safety surveillance objective; liveness surveillance objective}
\end{example}


\subsection{Incorporating Functional Objectives}
We can easily integrate functional LTL objectives not related to surveillance by considering in addition to $\SP$ a set $\AP$ of predicates interpreted over states of $G$. In order to define the semantics of $p \in \AP$ over states of $G_\belief$, we require for every $p \in \AP$ that if $(l_a,l_t')$, $(l_a,l_t'')$, and $\vis(l_a,l_t') = \vis(l_a,l_t'') = \false$, then $(l_a,l_t') \models p$ iff $(l_a,l_t'') \models p$. This restriction is without loss of generality, since or a finite set of predicates one can extend the state-space of $G$ by making the predicate valuations part of $L_a$.

\begin{example}
\todo{safety surveillance objective + liveness objective, liveness surveillance objective + liveness objective}
\end{example}


%%%%%%%%%%%%%%%%%%%%%%%%%%%%%%%%%%%%%%%%%%%%%%%%%%%%%%%%%%%%%%%%%%%%%%%%%%%%%%%%

\section{BELIEF SET ABSTRACTION}
%\subsection{Abstract Belief-Set Games}
We used the belief-set game structure in order to state the surveillance objective of the agent. While in principle it is possible to solve the surveillance strategy synthesis problem on this game, this is in most cases computationally infeasible, since the size of this game is exponential in the size of the original game. To circumvent this construction when possible, we propose an abstraction-based method, that given a surveillance game structure and a partition of the set of the target's locations, yields an approximation that is sound with respect to surveillance objectives for the agent.


An \emph{abstraction partition} is a family $\part = \{Q_i\}_{i=1}^n$ of subsets of $L_t$, $Q_i \subseteq L_t$ such that the following hold:
\begin{itemize}
\item $\bigcup_{i=1}^n Q_i = L_t$ and $Q_i \cap Q_j = \emptyset$ for all $i \neq j$;
\item For each $p \in \AP$, $Q \in \part$ and $l_a \in L_a$, it holds that $(l_a,l_t') \models p$ iff $(l_a,l_t'') \models p$ for every $l_t',l_t'' \in Q$.
\end{itemize}
Intuitively, these conditions mean that $\mathcal Q$ partitions the set of locations of the target, and the concrete locations in each of the sets in $\part$ agree on the value of the  propositions in $\AP$.

If $\part' =  \{Q_i'\}_{i=1}^m$ is a family of subsets of $L_t$ such that $\bigcup_{i=1}^m Q_i' = L_t$ and for each $Q_i' \in \part'$ there exists $Q_j \in \part$ such that $Q_i' \subseteq Q_j$, then $\part'$ is also an abstraction partition, and we say that $\part'$ \emph{refines} $\part$, denoted $\part' \preceq \part$.

For $\part = \{Q_i\}_{i=1}^n$,  we define a function $\alpha_\part : L_t \to \part$ by $\alpha(l_t) = Q$ for the unique $Q \in \part$ with $l_t \in Q$. We denote also with $\alpha_{\part} : \mathcal{P}(L_t) \to \mathcal{P}(\part)$ the \emph{abstraction function} defined by $\alpha_{\part}(L) = \{\alpha_\part(l) \mid l \in L\}$.
We define a \emph{concretization function} $\gamma :  \mathcal{P}(\part) \cup L_t \to \mathcal{P}(L_t)$ such that 
$\gamma(A) = \{l_t\}$ if $A = l_t \in L_t$, and  $\gamma(A) = \bigcup_{Q \in A} Q$ if $A \in \mathcal{P}(\part)$.
%$
%\gamma(A) =
%\begin{cases}
%\{l_t\} &\text{if } A = l_t \in L_t,\\
%\bigcup_{Q \in A} Q &\text{if } A \in \mathcal{P}(\mathcal{Q}).\\
%\end{cases}
%$

Given a surveillance game structure $G  = (\states,s^\init,\trans,\vis)$ and an abstraction partition $\part = \{Q_i\}_{i=1}^n$ of the set $L_t$, we define the \emph{abstraction of $G$ w.r.t.\ $\part$} to be the game structure 
$G_\abstr  = \alpha_{\part}(G)= (\states_\abstr,s^\init_\abstr,\trans_\abstr)$, where

\begin{itemize}
\item $\states_\abstr = (L_a \times \mathcal P(\part)) \cup (L_a \times L_t)$  is the set of \emph{abstract states}, consisting of states approximating the belief sets in the game structure $G_\belief$, as well as the states $\states$;
\item $s^\init_\abstr = (l_a^\init,l_t^\init)$ is the \emph{initial abstract state};
%, where we let $A_t^{\init} = l_t^\init$ if $\vis(l_a,l_t) = \true$, and we define $A_t^{\init} = \{\alpha_{\part}(l_t^\init)\}$ in case $\vis(l_a,l_t) = \false$.
\item $\trans_\abstr \subseteq \states_\abstr \times \states_\abstr$ is the transition relation such that $((l_a, A_t),(l_a', A_t')) \in \trans_\abstr$ if and only if one of the following two conditions is satisfied:
\begin{itemize}
\item[(1)] $A_t' = l_t'$, $l_t' \in \post(\gamma(A_t))$ and $\vis(l_a,l_t') = \true$, and
$l_a' \in \succs_a(l_a,l_t,l_t')$ for some $l_t \in \gamma(A_t)$.
\item[(2)] $A_t' = \alpha_{\part}(\{l_t' \in \post(\gamma(A_t))  |  \vis(l_a,l_t') = \false\})$, and
$l_a' \in \succs_a(l_a,l_t,l_t')$ for some $l_t \in \gamma(A_t)$ and some
$l_t' \in \post(\gamma(A_t))$ with $\vis(l_a,l_t') = \false$.
%\item[(1)] $A_t' = l_t'$ for some $l_t' \in L_t$, where $\vis(l_a,l_t') = \true$, and $((l_a,l_t),(l_a',l_t')) \in \trans$ for some $l_t \in \gamma(A_t)$;

%\item[(2)] $A_t' = \alpha_{\part}(B_t')$, where 

%$\begin{array}{lll}
%B_t' = \{l_t' & \mid & \vis(l_a,l_t') = \false \text{ and } \\
%&& \exists l_t \in \gamma(A_t).\ ((l_a,l_t),(l_a',l_t')) \in \trans\}.
%\end{array}
%$
\end{itemize}
\end{itemize}

As for the belief-set game structure, the first condition captures the successor locations of the target, which can be observed from the agent's current location $l_a$. Condition (2) corresponds to the \emph{abstract belief set} whose concretization  consists of all possible successors of all positions in $\gamma(A_t)$, which are  not visible from $l_a$. Since the belief abstraction overapproximates the agent's belief, that is, $\gamma(\alpha_{\part}(B)) \supseteq B$, the next-state abstract belief $\gamma(A_t')$ may include positions in $L_t$ that are not successors of positions in $\gamma(A_t)$.

%\begin{itemize}
%\item[(1)] $A_t' = l_t'$ for some $l_t' \in L_t$, where $\vis(l_a,l_t') = \true$, and $((l_a,l_t),(l_a',l_t')) \in \trans$ for some $l_t \in \gamma(A_t)$;
%\item[(2)] there exists $l_t \in \gamma(A_t)$ with $\vis(l_a,l_t) = \true$, and $A_t' = \alpha_{\part}(B_t')$, where
%
%$\begin{array}{lll}
%B_t' = \{l_t' & \mid & \vis(l_a,l_t') = \false \text{ and } \\
%&& ((l_a,l_t),(l_a',l_t')) \in \trans\};
%\end{array}
%$
%
%\item[(3)] $A_t' = \alpha_{\part}(B_t')$, where 
%
%$\begin{array}{lll}
%B_t' = \{l_t' & \mid & \vis(l_a,l_t') = \false \text{ and } \\
%&& \exists l_t \in \gamma(A_t): \vis(l_a,l_t') = \false \\
%&& \text{and }  ((l_a,l_t),(l_a',l_t')) \in \trans\}.
%\end{array}
%$
%\end{itemize}
%As for the belief-set game structure, the first condition captures of target that can be observed from the agent's current location $l_a$. Conditions (2) and (3) correspond to \emph{abstract} belief sets whose concretization includes all possible successor locations of the target not visible from $l_a$. In (2) those are successors of a single possible current position $l_t$ of the target that is visible from $l_a$, while in (3) the belief consist of  successors of all positions in $B_t$ not visible from $l_a$. Since the belief-abstraction overapproximates the agent's belief, that is, $\gamma(\alpha_{\part}(B)) \supseteq B$, the abstract belief $\gamma(A_t')$ may include positions that are not successors of positions in $\gamma(A_t)$.

\begin{figure}
\input{figs/simple-abstr-transitions.tex}
\vspace{-.2cm}
\caption{Transitions from the initial state in the abstract game from Example~\ref{ex:simple-abstr-game} where $\alpha_\part(17) = Q_4$ and $\alpha_\part(23) = Q_5$.}
\label{fig:simple-abstr-game}
\vspace{-.7cm}
\end{figure}


\begin{example}\label{ex:simple-abstr-game}
Consider again the surveillance game from Example~\ref{ex:simple-surveillance-game}, and the abstraction partition $\part = \{Q_1,\ldots,Q_5\}$, where the set $Q_i$ corresponds to the $i$-th row of the grid. For location $17$ of the target we have $\alpha_\part(17) = Q_4$, and for  $23$ we have $\alpha_\part(23) = Q_5$. Thus, the belief set $B = \{17,23\}$ is covered by the abstract belief set $\alpha_Q(B) = \{Q_4,Q_5\}$. Figure~\ref{fig:simple-abstr-game} shows the successors of the initial state $(4,18)$ of the abstract belief-set game structure. The concretization of the abstract belief set $\{Q_4,Q_5\}$ is the set $\{15,16,17,18,19,20,21,22,23,24\}$ of target locations.\qed
\end{example}

An abstract state $(l_a,A_t)$ \emph{satisfies a surveillance predicate $p_k$}, denoted $(l_a,A_t) \models p_k$, iff 
$|\{l_t \in \gamma(A_t) \mid \vis(l_a,l_t)  = \false \}| \leq k$. Simply, the number of states in the concretized belief set has to be less than or equal to $k$. Similarly, for a predicate $p \in \AP$, we define $(l_a,A_t) \models p$ iff for every $l_t \in \gamma(A_t)$ it holds that $(l_a,l_t) \models p$. With these definitions, we can interpret surveillance objectives over runs of $G_\abstr$.

Strategies (and wining strategies) for the agent and the target in an abstract belief-set game $(\alpha_\part(G),\varphi)$ are defined analogously to strategies (and winning strategies) in $G_\belief$.

%\subsection{Soundness of Belief Set Abstraction}
In the construction of the abstract  game structure, we overapproximate the belief-set of the agent at each step. Since we consider surveillance predicates that impose upper bounds on the size of the belief, such an abstraction  gives more power to the target (and, dually less power to the agent).  This construction guarantees that the abstraction is \emph{sound}, meaning that an abstract strategy for the agent that achieves a surveillance objective corresponds to a winning strategy in the concrete game. This is stated in the following theorem.

\begin{theorem}
Let $G$ be a surveillance game structure, $\part = \{Q_i\}_{i=1}^n$ be an abstraction partition, and $G_\abstr = \alpha_\part(G)$. For every surveillance objective $\varphi$, if there exists a wining strategy for the agent in the abstract belief-set game $(\alpha_\part(G),\varphi)$, then there exists a winning strategy for the agent in the concrete surveillance game $(G,\varphi)$.
\end{theorem}



%%%%%%%%%%%%%%%%%%%%%%%%%%%%%%%%%%%%%%%%%%%%%%%%%%%%%%%%%%%%%%%%%%%%%%%%%%%%%%%%

\section{BELIEF REFINEMENT FOR SAFETY}% SURVEILLANCE OBJECTIVES}
\subsection{Counterexample Tree}
A winning strategy for the target in a game with safety surveillance objective can be represented as a tree. 
An \emph{abstract counterexample tree} $\counterex_\abstr$ for $(G_\abstr,\LTLglobally p_k)$ is a finite tree,  whose nodes are labelled with states in $\states_\abstr$ such that the following conditions are satisfied:
\begin{itemize}
\item The root node is labelled with the initial state $s_\abstr^\init$.
\item A node is labelled with an abstract state  which violates $p_k$ (that is, $s_\abstr$ where $s_\abstr \not\models p_k$) iff it is a leaf.
\item The tree branches according to all possible transition choices of the agent. Formally, if an internal node $v$ is labelled with $(l_a,A_t)$, then there is unique $A_t'$  such that: (1) $((l_a,A_t),(l_a',A_t')) \in \trans_\abstr$ for some $l_a' \in L_a$, and (2) for every $l_a' \in L_a$ such that $((l_a,A_t),(l_a',A_t')) \in \trans_\abstr$, there is a child $v'$ of $v$ labelled with $(l_a',A_t')$.
\end{itemize}


A \emph{concrete counterexample tree} $\counterex_\belief$ for $(G_\belief,\LTLglobally p_k)$ is a finite tree defined analogously to an abstract counterexample tree with nodes labelled with states in $\states_\belief$.
% where:
%\begin{itemize}
%\item The root node is labelled with the initial state $s_\belief^\init$;
%\item A node is labelled with a belief state which violates $p_k$ (that is, $s_\belief$ where $s_\belief \not\models p_k$) iff it is a leaf;
%\item The tree branches according to all possible transition choices of the agent. Formally, if an internal node $v$ is labelled with $(l_a,B_t)$, then there exists a $B_t'$  such that: (1) $((l_a,B_t),(l_a',B_t')) \in \trans_\belief$ for some $l_a' \in L_a$, and (2) for every $l_a' \in L_a$ such that $((l_a,B_t),(l_a,B_t')) \in \trans_\belief$, there is a child $v'$ of $v$ labelled with $(l_a',B_t')$.
%\end{itemize}

Due to the overapproximation of the belief sets, not every counterexample in the abstract game corresponds to a winning strategy for the target in the original game.

An abstract counterexample $\counterex_\abstr$ in $(G_\abstr,\LTLglobally p_k)$ is \emph{concretizable} if there exists a concrete counterexample 
tree $\counterex_\belief$ in $(G_\belief,\LTLglobally p_k)$, that differs from $\counterex_\abstr$ only in the node labels, and each node labelled with $(l_a,A_t)$ in $\counterex_\abstr$ has label $(l_a, B_t)$ in $\counterex_\belief$ for which $B_t \subseteq \gamma(A_t)$.


\begin{figure}
\subfloat[Abstract counterexample tree\label{fig:simple-safety-counterex-abstr}]{
\input{figs/simple-safety-counterex-abstr.tex}\hspace{.5cm}
}
\hfill
\subfloat[Concrete counterexample tree\label{fig:simple-safety-counterex-concr}]{
\begin{tikzpicture}[node distance=.9 cm,auto,>=latex',line join=bevel,transform shape,scale=.8]
\node at (0,0) (s0) {$(12,32)$};
\node  [below left of=s0,yshift=-.5cm,xshift=-.5cm] (s1) {$(11,\{31,39\})$};
\node  [below right of=s0,yshift=-.5cm,xshift=.5cm] (s2) {$(19,\{31,39\})$};

\draw [->] (s0) edge (s1.north);
\draw [->] (s0) edge (s2.north);
\end{tikzpicture}
\hspace{.5cm}
}
\caption{Abstract and corresponding concrete counterexample trees for the surveillance game in Example~\ref{ex:simple-safety-counterex}.}
\label{fig:simple-safety-counterex}
\vspace{-.5cm}
\end{figure}

\begin{example}\label{ex:simple-safety-counterex}
Figure~\ref{fig:simple-safety-counterex-abstr} shows an abstract counterexample tree $\counterex_\abstr$ for the game $(\alpha_\part(G),\LTLglobally p_1)$, where $G$ is the surveillance game structure from Example~\ref{ex:simple-surveillance-game} and $\part$ is the abstraction partition from Example~\ref{ex:simple-abstr-game}. The counterexample corresponds to the choice of the target to move to one of the locations $17$ or $23$, which, for every possible move of the agent, results in an abstract state with abstract belief $B = \{Q_4,Q_5\}$ violating $p_1$.
A concrete counterexample tree $\counterex_\belief$ concretizing $\counterex_\abstr$ is shown in Figure~\ref{fig:simple-safety-counterex-concr}.
\qed
\end{example}

\subsection{Counterexample-Guided Refinement}
\subsubsection{Forward belief-set propagation}
Given an abstract counterexample tree $\counterex_\abstr$ we label its nodes with states in $\states_\belief$ in a top-down manner as follows. 
The root node is labelled with $s_\belief^\init$. If a node $v$ is a node labelled with the belief set $(l_a,B_t) \in \states_\belief$, and  $v'$ is a child of $v$ in $\counterex_\abstr$ labelled with an abstract state $(l_a',A_t')$, then we label $v'$ with the belief set $(l_a',B_t')$, where 
$B_t' = \post(B_t) \cap \gamma(A_t')$. The counterexample analysis procedure based on this annotation is given in Algorithm~\ref{algo:cex-analysis}.
If each of the leaf nodes of the tree is annotated with a belief set $(l_a,B_t)$ for which $(l_a,B_t) \not\models p_k$, then the new annotation gives us a concrete counterexample tree $\counterex_\belief$, which by construction concertizes $\counterex_\abstr$. Conversely, if there exists a leaf node labelled with $(l_a,B_t)$ such that $(l_a,B_t) \models p_k$, then we can conclude that the abstract counterexample tree $\counterex_\abstr$ is not concretizable and use the path from the root of the tree to this leaf node to refine the partition $\part$.

\begin{theorem}\todo{correctness of counterexample analysis}
\end{theorem}

\subsubsection{Backward partition splitting}
Consider a path $\pi_\abstr = v_0,\ldots, v_n$ in $\counterex_\abstr$ where $v_0$ is the root node and $v_n$ is a leaf. For each node $n_i$, let $(l_a^i,A_t^i) $ be the abstract state labelling $n_i$ in $\counterex_\abstr$, and let $(l_a^i,B_t^i)$ be the  belief set with which the node was labelled by the the counterexample analysis procedure. We consider the case when $(l_a^n,B_t^n) \models p_k$, that is, $|\{l_t \in B_t^n \mid \vis(l_a,l_t) = \false\}| \leq k$.
Note that since $\counterex_\abstr$ is a counterexample we have $(l_a^n,A_t^n) \not \models p_k$, and since $k>0$, this means $A_t \subseteq \mathcal{P}(\mathcal Q)$.


We now describe a procedure to compute a partition $\part'$ that refines the current partition $\part$ based on the path $\pi_\abstr$. Intuitively, we split the partition states that appear in $A_t^n$ with the goal to ensure that in the refined abstract game the corresponding abstract state satisfies the surveillance predicate $p_k$. To achieve this, however, we might have to also split partition elements appearing in abstract states on the path to $n_b$. The reason is that we have to ensure that earlier imprecisions on this path do not propagate and including more of the newly split partition states leading to the same violation of $p_k$.
Formally, if $A_t^n = (l_a^n,\{B_{n,1},\ldots,B_{n,m_n}\})$, then we split some of the partition states $B_{n,1},\ldots,B_{n,m_n}$ to obtain from $A_t^n$ a set $A_j' = \{B_{n,1}',\ldots,B_{n,m_n'}'\}$ such that
\[|\{l_t \in \gamma(C^n) \mid \vis(l_a^n,l_t) = \false\}| \leq k \text{, where}\] 
\[C^n = \{B_{n,i}' \in A_j' \mid B_{n,i}' \cap B_t^n \neq \emptyset\}.\]
This property intuitively means that if we consider the sets in $A'$ that have non-empty intersection with $B_t^n$, an abstract state composed of those partition sets will satisfy $p_k$. Since $(l_a^n,B_t^n)$ satisfies $p_k$, we can find a partition $\part^n \preceq \part$ that guarantees this property. What remains in order to eliminate this counterexample, is to ensure that only these partition states are reachable via the considered path, by propagating this splitting backwards, obtaining a sequence of partitions $\part \succeq \part^n \succeq \part^{n-1} \succeq \ldots \succeq \part^0$ refining $\part$. Given $\part^{j+1}$, we compute $Q^j$ as follows. For each $j$, we define a set $C^j \subseteq \mathcal{P}(L_t)$ (for $j=n$, the set $C^n$ was defined above). Suppose we have defined $C^{j+1}$ for some $j \geq 0$, and $A_t^j = (l_a^j,\{B_{j,1},\ldots,B_{j,m_j}\})$. We split some of the sets $B_{j,1},\ldots,B_{j,m_j}$ to obtain from $A_t^j$ a set $A_j' = \{B_{j,1}',\ldots,B_{j,m_j'}'\}$ where there exists $C^j \subseteq A_j'$ with
\[\gamma(C^j) = \gamma(A_t^j) \cap \{l_t \mid \post(l_a^j,l_t) \cap \gamma(A_t^{j+1}) \subseteq \gamma(C^{j+1})\}.\] 
Intuitively, this means that the new partition allow us to express precisely the set of states that do not lead to sets in $A_{j+1}'$ that we are trying to avoid. Again, the fact that an appropriate partition $\part$ can be computed follows from the choice of the leaf node $v_n$. The procedure for computing the partition $\part' = \part^0$ that refines $\part$ based on such a path $\pi_\abstr$ is formalized in Algorithm~\ref{}, and the theorem below states the progress property (eliminating the considered counterexample) which it guarantees.


\begin{theorem}\todo{progress property of the refinement procedure: counterexample eliminated}
\end{theorem}

\begin{example}
\todo{illustrating example}
\end{example}


%%%%%%%%%%%%%%%%%%%%%%%%%%%%%%%%%%%%%%%%%%%%%%%%%%%%%%%%%%%%%%%%%%%%%%%%%%%%%%%%

\section{BELIEF REFINEMENT FOR LIVENESS}% SURVEILLANCE OBJECTIVES}
\subsection{Counterexample Graph}
The counterexamples for general surveillance properties are directed graphs, which may contain cycles. In particular, for a liveness surveillance property of the form $\LTLglobally\LTLfinally p_k$ each infinite path in the graph has a position such that, from this position on, each state on the path violates $p_k$. An \emph{abstract counterexample graph} in the game $(G_\abstr,\LTLglobally\LTLfinally p_k)$ is a finite graph $\counterex_\abstr$ defined analogously to the abstract counterexample tree. The difference being that there are no leaves, and that for each cycle $\rho = v_1,v_2,\ldots,v_n$ with $v_1 = v_n$ in $\counterex_\abstr$ that is reachable from $v_0$, every node $v_i$ in $\rho$ is labelled with state $s_\abstr^i$ where $s_\abstr^i \not\models p_k$.
%\begin{itemize}
%\item There exists a node $v_0$ of $\counterex_\abstr$ labelled with $s^\init_\abstr$.
%\item For each cycle $\rho = v_1,v_2,\ldots,v_n$ with $v_1 = v_n$ in $\counterex_\abstr$ that is reachable from $v_0$, every node $v_i$ in $\rho$ is labelled with state $s_\abstr^i$ where $s_\abstr^i \not\models p_k$.
%\item The graph branches according to all possible transition choices of the agent. Formally, if a node $v$ is labelled with $(l_a,A_t)$, then there exists an $A_t'$  such that: (1) $((l_a,A_t),(l_a',A_t')) \in \trans_\abstr$ for some $l_a' \in L_a$, and (2) for every $l_a' \in L_a$ such that $((l_a,A_t),(l_a,A_t')) \in \trans_\abstr$, there is a child $v'$ of $v$ labelled with $(l_a',A_t')$.
%\end{itemize}   

\begin{figure}
\begin{minipage}{0.2\textwidth}
\begin{center}
%\includegraphics[scale=.33]{figs/7x7_liveness.png}
\begin{tikzpicture}[scale=0.9]
\draw[step=0.5cm,color=gray] (-1.5,-1.5) grid (1,1);
\filldraw[fill=blue,draw=black] (+0.75,+0.75) circle (0.2cm);
\filldraw[fill=red,draw=black] (0,0) rectangle (-0.5,-0.5);
\filldraw[fill=red,draw=black] (-0.5,0) rectangle (-1,-0.5);
\filldraw[fill=red,draw=black] (0,0) rectangle (0.5,-0.5);
\filldraw[fill=blue!40!white,draw=black] (+0.75,+0.75) circle (0.2cm);
\filldraw[fill=orange!40!white,draw=black] (0.25,-0.75) circle (0.2cm);
\draw[blue,thick] (-1.35,-0.75) rectangle (0.75,0.25);
\draw[blue,thick,->] (0.73,0.75) -> (0.75,0.25);
\node at (-1.30,+0.75) {\tiny{0}};
\node at (-0.80,+0.75) {\tiny{1}};
\node at (-0.30,+0.75) {\tiny{2}};
\node at (0.20,+0.75) {\tiny{3}};
\node at (0.73,+0.75) {\tiny{4}};
\node at (-1.35,+0.25) {\tiny{5}};
\node at (-0.85,+0.25) {\tiny{6}};
\node at (-0.35,+0.25) {\tiny{7}};
\node at (0.25,+0.25) {\tiny{8}};
\node at (0.75,+0.25) {\tiny{9}};
\node at (-1.35,-0.25) {\tiny{10}};
\node at (-0.85,-0.25) {\tiny{11}};
\node at (-0.35,-0.25) {\tiny{12}};
\node at (0.25,-0.25) {\tiny{13}};
\node at (0.75,-0.25) {\tiny{14}};
\node at (-1.35,-0.75) {\tiny{15}};
\node at (-0.85,-0.75) {\tiny{16}};
\node at (-0.35,-0.75) {\tiny{17}};
\node at (0.25,-0.75) {\tiny{18}};
\node at (0.75,-0.75) {\tiny{19}};
\node at (-1.35,-1.25) {\tiny{20}};
\node at (-0.85,-1.25) {\tiny{21}};
\node at (-0.35,-1.25) {\tiny{22}};
\node at (0.25,-1.25) {\tiny{23}};
\node at (0.75,-1.25) {\tiny{24}};
\end{tikzpicture}
\end{center}
\end{minipage}
\begin{minipage}{0.28\textwidth}
\vspace{0.1cm}
\caption{\small Agent locations on an (infinite) path in the abstract counterexample graph from Example~\ref{ex:simple-liveness-counterex}. In the graph, the first node is labelled with $(4,18)$, the second with $(9,\{Q_2\})$, and all other nodes with some $(l_a,\{Q_1,Q_2\})$.}
\label{fig:simple-liveness-counterex}
\end{minipage}
\vspace{-.65cm}
\end{figure}

\begin{example}\label{ex:simple-liveness-counterex}
We saw in Example~\ref{ex:simple-safety-realizability} that in the safety surveillance game $(G,\LTLglobally p_2)$ the agent does not have a winning strategy. %This means, that the agent cannot ensure keeping at all times the uncertainty about the current position of the target to at most two positions.
We now consider a relaxed requirement, namely, that the uncertainty drops to at most $2$ infinitely often. We consider the liveness surveillance game 
$(G,\LTLglobally \LTLfinally p_2)$.

Let $\part = \{Q_1,Q_2\}$ be the partition from Example~\ref{ex:simple-safety-unconcretizable}. %that is, $Q_1$, corresponds to the first two columns of the grid in Figure~\ref{simple-grid} and the set $Q_2$ contains the locations from the other three columns of the grid. 
Figure~\ref{fig:simple-liveness-counterex} shows an infinite path (in lasso form) in the abstract game $(\alpha_\part(G),\LTLglobally \LTLfinally p_2)$.  The figure depicts only the corresponding trajectory (sequence of positions) of the agent. The initial abstract state is $(4,18)$, the second node on the path is labeled with the abstract state $(9,\{Q_2\})$, and all other nodes on the path are labeled with abstract states of the form $(l_a,\{Q_1,Q_2\})$. As each abstract state in the cycle violates $p_2$, the path violates $\LTLglobally \LTLfinally p_2$. The same holds for all infinite paths in the existing abstract counterexample graph.
\qed
\end{example}

A \emph{concrete counterexample graph} $\counterex_\belief$ for the belief game $(G_\belief,\LTLglobally\LTLfinally p_k)$ is defined analogously. 

An abstract counterexample graph $\counterex_\abstr$ for the game $(G_\abstr,\LTLglobally\LTLfinally p_k)$ is \emph{concretizable} if there exists a counterexample
$\counterex_\belief$ in $(G_\belief,\LTLglobally \LTLfinally p_k)$, such that for each infinite path $\pi_\abstr = v_\abstr^0,v_\abstr^1,\ldots$ starting from the initial node of $\counterex_\abstr$ there exists an infinite path $\pi_\belief = v_\belief^0,v_\belief^1,\ldots$ in $\counterex_\belief$ staring from its initial node such that if $v_\abstr^i$ is labelled with $(l_a,A_t)$ in $\counterex_\abstr$, then the corresponding node $v_\belief^i$ in $\counterex_\belief$ is labelled with $(l_a,B_t)$ for some $B_t \in \mathcal{P}(L_t)$ for which $B_t \subseteq \gamma(A_t)$.


\subsection{Counterexample-Guided Refinement}
\subsubsection{Forward belief-set propagation}

To check if an abstract counterexample graph $\counterex_\abstr$ is concretizable, we construct a finite graph $\mathcal{D}$ whose nodes are labelled with elements of $\states_\belief$ and with nodes of $\counterex_\abstr$.
By construction we will ensure that if a node $d$ in $\mathcal D$ is labelled with $\langle(l_a,B_t),v \rangle$, where $(l_a,B_t)$ is a belief state, and $v$ is a node in $\counterex_\abstr$, then $v$ is labelled with $(l_a,A_t)$ in $\counterex_\abstr$, and $B_t \subseteq \gamma(A_t)$. 

Initially $\mathcal D$ contains a single node $d_0$ labelled with $\langle s_\belief^\init,v_0\rangle$, where $v_0$ is initial node of $\counterex_\abstr$. Consider a node $d$ in $\mathcal D$ labelled with $\langle(l_a,B_t),v \rangle$. For every child $v'$ of $v$ in $\counterex_\abstr$, labelled with an abstract state $(l_a',A_t')$ we proceed as follows. We let ${B_t}' = \post(l_a,B_t) \cap \gamma(A_t')$. If there exists a node $d'$ in $\mathcal D$ labelled with $\langle (l_a',B_t'),v\rangle$, then we add an edge from $d$ to $d'$ in $\mathcal{D}$. Otherwise, we create such a node and add the edge. We continue until no more nodes and edges can be added to $\mathcal D$. The procedure is guaranteed to terminate, since both  the graph $\counterex_\belief$, and $\states_\belief$ are finite, and we add a node labelled $\langle s_\belief, v\rangle$ to $\mathcal D$ at most once.

If the graph $\mathcal D$ contains a reachable cycle (it suffices to consider simple cycles, i.e., without repeating intermediate nodes) $\rho = d_0,\ldots,d_n$ with $d_0 = d_n$ such that some $d_i$ is labelled with $(l_a,B_t)$ where $(l_a,B_t) \models p_k$, then we conclude that the abstract counterexample $\counterex_\abstr$ is not concretizable. If no such cycle exists, then $\mathcal D$ is a concrete counterexample graph for the belief game $(G_\belief,\LTLglobally\LTLfinally p_k)$. 

%\comment{
\begin{algorithm}[b] \small
\vspace{-.5cm}
\KwIn{surveillance game $(G,\LTLglobally\LTLfinally p_k)$, abstract counterexample graph $\counterex_\abstr$ with initial node $v_0$}
\KwOut{a path $\pi$ in a graph $\mathcal D$ or {\sc concretizable}}

\smallskip

graph $\mathcal D = (D,E)$ with nodes $D := \{d_0\}$ and edges $E := \emptyset$\;

annotate $d_0$ with $\langle s_\belief^\init, v_0\rangle$\; 

\SetKwRepeat{Do}{do}{while}

\Do{$\mathcal D \neq \mathcal D'$}{
$\mathcal D' := \mathcal D$\;
 \ForEach{node $d$ in $\mathcal D$ labelled with $\langle(l_a,B_t),v\rangle$}{
  \ForEach{child $v'$ of $v$ in $\counterex_\abstr$ labelled $(l_a',A_t')$}{
  $B_t' := \post(l_a,B_t)\cap\gamma(A_t')$\;
  \eIf{there is a node $d' \in D$ labelled with $\langle (l_a',B_t'),v'\rangle$}{add an edge $(d,d')$ to $E$}
  {add a node $d'$ labelled $\langle (l_a',B_t'),v'\rangle$ to $D$\;
  add an edge $(d,d')$ to $E$}
}
}
}
\leIf{there is a lasso path $\pi$ in $\mathcal D$ starting from $d_0$ such that some node in the cycle is annotated with $\langle s_\belief,v\rangle$ and $s_\belief\models p_k$\newline}
{\KwRet{$\pi$;}}
{\KwRet{{\sc concretizable}}}

\smallskip

\caption{Analysis of abstract counterexample graphs for games with liveness surveillance objectives.}
\label{algo:cex-analysis-liveness}
\end{algorithm}
%}

\begin{example}\label{ex:simple-liveness-unconcretizable}
The abstract counterexample graph in the game $(\alpha_\part(G),\LTLglobally \LTLfinally p_2)$ discussed in Example~\ref{ex:simple-liveness-counterex} is not conretizable, since for the path in the abstract graph depicted in Figure~\ref{fig:simple-liveness-counterex} there exists a corresponding path in the graph $\mathcal D$ with a node in the cycle labelled with a set in $G_\belief$ that satisfies $p_2$. More precisely, the cycle in the graph $\mathcal D$ contains a node labelled with $(19,\{10\})$. Intuitively, as the agent moves from the upper to the lower part of the grid along this path, upon not observing the target, it can infer from the sequence of observations that the only possible location of the target is $10$. Thus, this paths is winning for the agent.
\qed
\end{example}

\begin{theorem}
If Algorithm~\ref{algo:cex-analysis-liveness} returns a path $\pi$ in the graph $\mathcal D$ constructed for $\counterex_\abstr$, then $\counterex_\abstr$ is not concretizable, and the infinite run in $G_\belief$ corresponding to $\pi$ satisfies $\LTLglobally\LTLfinally p_k$, otherwise  $\counterex_\abstr$ is concretizable.
\end{theorem}

\subsubsection{Backward partition splitting}

Consider a path in the graph $\mathcal{D}$ of the form $\pi = d_0,\ldots, d_n,d_0',\ldots,d_m'$ where $d_n = d_m'$, and where for some $0 \leq i \leq m$ for the label $(l_a^i,B_t^i)$ it holds that $(l_a^i,B_t^i) \models p_k$. Let 
$\pi_\abstr = v_0,\ldots, v_n,v_0',\ldots,v_m'$ be the sequence of nodes in $\counterex_\abstr$ corresponding to the labels in $\pi$. By construction of $\mathcal D$, $\pi_\abstr$ is a path in $\counterex_\abstr$ and $v_n = v_m'$. We apply the refinement procedure from the previous section to the whole path $\pi_\abstr$, as well as to the path-prefix $v_0,\ldots, v_n$.

Let $\part$ and $\part'$ be two counterexample partitions such that $\part' \preceq \part$. Let $\counterex_\abstr$ be an abstract counterexample graph in $(\alpha_\part(G),\LTLglobally\LTLfinally p_k)$. We define $\gamma_{\part'}(\counterex_\abstr)$ to be the set of abstract counterexample graphs in $(\alpha_{\part'}(G),\LTLglobally\LTLfinally p_k)$ such that for every infinite path $\pi$ in $\counterex_\abstr'$ there exists an infinite path $\pi$ in $\counterex_\abstr$ such that for every node in $\pi'$ labelled with $(l_a,A_t')$ the corresponding node in $\counterex_\abstr$ is labelled with an abstract state $(l_a,A_t)$ such that $\gamma(A_t') \subseteq \gamma(A_t)$.

\begin{theorem}If $\part'$ is the partition 
%returned by Algorithm~\ref{algo:refinement-liveness} 
obtained by refining $\part$ with respect to an uncocretizable abstract counterexample $\counterex_\abstr$ in $(\alpha_\part(G),\LTLglobally\LTLfinally p_k)$, then $\gamma_{\part'}(\counterex_\abstr) = \emptyset$, and also $\gamma_{\part''}(\counterex_\abstr) = \emptyset$ for every partition $\part''$ with $\part'' \preceq \part'$.
\end{theorem}

\begin{example}\label{ex:simple-liveness-refinement}
We refine the abstraction partition $\part$ from Example~\ref{fig:simple-liveness-counterex} using the path identified there, in order to eliminate the abstract counterexample. For this, following the refinement algorithm, we first split the set $Q_1$ into sets $Q_1' = \{10\}$ and $Q_2' = Q_1 \setminus \{10\}$, and let $Q_3' = Q_2$. However, since from some locations in $Q_2'$ and in $Q_3'$ the target can reach locations in $Q_2'$ and $Q_3'$ that are not visible from the agent's position $19$, in order to eliminate the counterexample, we need to propagate the refinement backwards along the path and split $Q_2'$ and $Q_3'$ further. With that, we obtain an abstraction partition with $10$ sets, which is guaranteed to eliminate this abstract counterexample. In fact, in this example this abstraction turns out to be sufficiently precise to obtain a winning strategy for the agent.
\qed
\end{example}

\subsection{General surveillance and task specifications}
\todo{omega-regular surveillance and task}
\begin{itemize}
\item $\LTLglobally p_k \wedge \LTLglobally\LTLfinally p_l$: first check concretizability of all paths in the graph to states violating the safety constraints, if one is not concretizable then done, if all are concretizable, then apply method for liveness
\item general surveillance objectives: refine for some node where the true belief is more precise than the abstract one. not guaranteed to eliminate counterexample, but refinement loop guaranteed to still terminate since system is finite state and a proper split is done at each step. heuristics for identifying such nodes
\item $\varphi_{\mathit{surveillance}} \wedge \varphi_{\mathit{task}}$  if surveillance is a conjunction of safety and liveness, then in sequence: safety, liveness, task; otherwise as previous item
\item full specification language: as for general surveillance
\end{itemize}

\todo{practical fragment GR(1)}: used in implementation

%%%%%%%%%%%%%%%%%%%%%%%%%%%%%%%%%%%%%%%%%%%%%%%%%%%%%%%%%%%%%%%%%%%%%%%%%%%%%%%%

\section{EXPERIMENTAL EVALUATION}\label{sec:experiments}
We now return to the case study outlined in Section~\ref{sec:casestudy}. We have implemented the proposed method in \texttt{Python}, using the \texttt{slugs} reactive synthesis tool~\cite{EhlersR16}, and evaluated it on the multi-agent surveillance game modelling the problem described in Section~\ref{sec:casestudy}. The experiments were performed on an Intel i5-5300U 2.30 GHz CPU with 8 GB of RAM.

 We analyzed two scenarios. In Figure \ref{fig:experiment}, we have six \emph{mobile sensors}. We compare the surveillance strategy with the situation in Figure \ref{fig:3experiment} where we have three \emph{mobile sensors}. In both cases there are four \emph{static sensors} depicted in yellow in Figure~\ref{fig:bigexp}. Our global surveillance task is $\LTLsquare \LTLdiamond p_5$, i.e, we need to infinitely often bring the belief of the target location to 5 cells or lower. 

\begin{figure}
	\centering
\subfloat[The gridworld in \ref{fig:SGR-grid}\newline partitioned into 6 subgames. \label{fig:experiment}]{
\includegraphics[scale=0.18]{figs/SGR-grid-vis-part.png}
\hspace{1.5cm}}
%\hfill
\subfloat[The gridworld in \ref{fig:SGR-grid}\newline partitioned into 3 subgames. \label{fig:3experiment}]{
\includegraphics[scale=0.18]{figs/SGR-grid-vis-part_3.png}
\hspace{.3cm}}

\caption{Cases with 6 mobile sensors in Fig \ref{fig:experiment} and 3 mobile sensors in Fig \ref{fig:3experiment}. The mobile sensors are blue circles and the target is represented in orange. Yellow regions represent static sensors.The red cells represent impassable terrain (such as dense foliage) that cannot be seen through by the sensors. Black cells are locations not visible to any sensor.}\label{fig:bigexp}\vspace{-0.5cm}
\end{figure} 

Solving either case centralized is not computationally feasible as the state space grows exponentially with the number of sensors - we will have in the order of $400^6$ and $400^3$ states respectively. Thus, we partition the multi-agent surveillance game into subgames as shown in Figures \ref{fig:experiment} and \ref{fig:3experiment}. We then solve each game individually with local specifications $\locspec_i(\LTLglobally\LTLfinally p_5)$. We solve these \emph{single-agent} surveillance games using an abstraction-based method detailed in a companion publication at CDC 2018, detailed in~\cite{arxiv}. 
We report the synthesis times in Table \ref{tab:synthtime}.

\begin{table}[h!]
\vspace{-0.2cm}
	\centering
	\caption{Synthesis times for each surveillance subgame}
	\label{tab:synthtime}
	\begin{tabular}{c|l|l|l}
		\multicolumn{1}{l|}{}                                    & \textbf{Subgame} & \textbf{Number of states} & \textbf{Synthesis time (s)} \\ \hline \hline
		\multirow{7}{*}{\textbf{6 sensors}}
		& Subgame 1   & 69     & 101                          \\
	    & Subgame 2   & 74     & 206                          \\
		& Subgame 3   & 62     & 111                          \\
		& Subgame 4   & 52     & 88                          \\
		& Subgame 5   & 77     & 285                          \\
		& Subgame 6   & 66     & 64                          \\ \hline
		& \textbf{Total}   & \textbf{400}         & \textbf{855}                         \\ \hline
		\multicolumn{1}{l|}{\multirow{4}{*}{\textbf{3 sensors}}} & Subgame 1        & 142 & 473                         \\
		\multicolumn{1}{l|}{}                                    & Subgame 2        & 113 & 306                         \\
		\multicolumn{1}{l|}{}                                    & Subgame 3        & 145 & 372                         \\ \hline
		\multicolumn{1}{l|}{}                                    &  \textbf{Total} & \textbf{400}            & \textbf{1151}                        
	\end{tabular}
\end{table}
\vspace{-0.2cm}
The multi-agent surveillance game in Figure \ref{fig:experiment} results in more subgames compared to the game in \ref{fig:3experiment}. However, each game is much smaller and strategies can be synthesized faster in each subgame. Figure \ref{fig:case1exp} shows snapshots in time of the simulation of the 3 sensor surveillance game in Figure \ref{fig:3experiment}. The target is being controlled by a human and the sensors are following their synthesized local surveillance strategies. The global belief is depicted in Figure \ref{fig:case1exp} as grey cells, meaning that the combined knowledge of all the sensors has restricted the location of the target into one of the grey cells.
\begin{figure}
	
	\begin{minipage}{5.0cm}
		\centering
		\subfloat[$t_8$ \label{fig:case1t2}]{
			\includegraphics[scale=0.12]{figs/results_t2.png}\hspace{.7cm}
		}
		\subfloat[$t_{12}$ \label{fig:case1t3}]{
			\includegraphics[scale=0.12]{figs/results_t3.png}\hspace{.7cm}
		}
		\subfloat[$t_{16}$ \label{fig:case1t4}]{
			\includegraphics[scale=0.12]{figs/results_t4.png}\hspace{.7cm}
		}
	\end{minipage}
	\begin{minipage}{5.0cm}
		\centering
		\subfloat[$t_{18}$  \label{fig:case1t5}]{
			\includegraphics[scale=0.12]{figs/results_t1.png}\hspace{.7cm}
		}
		\subfloat[$t_{20}$ \label{fig:case1t6}]{
			\includegraphics[scale=0.12]{figs/results_t5.png}\hspace{.7cm}
		}
		\subfloat[$t_{22}$ \label{fig:case1t7}]{
			\includegraphics[scale=0.12]{figs/results_t6.png}\hspace{.7cm}
		}
		
	\end{minipage}
	
	
	\caption{Figures \ref{fig:case1t2} - \ref{fig:case1t7} are chronological snapshots during simulation of the surveillance game in Figure \ref{fig:3experiment}. Grey regions represent the global belief of the target's location.  
	}\vspace{-0.5cm}
	\label{fig:case1exp}
	
\end{figure}

 We see, in Figures \ref{fig:case1t2} - \ref{fig:case1t4}, that the target is in the subgame corresponding to sensor 2. Hence, only sensor 2 is moving and trying to lower its belief to below 5 cells (which it does in Figure \ref{fig:case1t5}). In Figures \ref{fig:case1t3} - \ref{fig:case1t5}, the target starts moving towards subgame 3 at which point the target is detected by the static sensor in subgame 3 and sensor 3 takes over in figures \ref{fig:case1t6} - \ref{fig:case1t7}. There is no communication between any of the agents, and each satisfy only their local surveillance specification. However, our construction guarantees that the global specification of $\LTLfinally \LTLglobally p_5$ will be satisfied. %We include a link to a video of the simulation at \url{google.com}
 
 %In figure \ref{fig:case1t2}, the target is in the region of a static sensor. Hence, the global belief is restricted to the region that sensor operates. 

%%%%%%%%%%%%%%%%%%%%%%%%%%%%%%%%%%%%%%%%%%%%%%%%%%%%%%%%%%%%%%%%%%%%%%%%%%%%%%%%

%\section{CONCLUSIONS}
%Future work
\begin{itemize}
\item incremental game solving: use result from previous iterations
\item first safety surveillance: most general strategy (smaller state space), then liveness
\item multiple targets
\end{itemize}

%%%%%%%%%%%%%%%%%%%%%%%%%%%%%%%%%%%%%%%%%%%%%%%%%%%%%%%%%%%%%%%%%%%%%%%%%%%%%%%%

%\section*{APPENDIX}

%%%%%%%%%%%%%%%%%%%%%%%%%%%%%%%%%%%%%%%%%%%%%%%%%%%%%%%%%%%%%%%%%%%%%%%%%%%%%%%%

%\section*{ACKNOWLEDGMENT}


%%%%%%%%%%%%%%%%%%%%%%%%%%%%%%%%%%%%%%%%%%%%%%%%%%%%%%%%%%%%%%%%%%%%%%%%%%%%%%%%
\bibliographystyle{IEEEtran}
\bibliography{main}

\end{document}
