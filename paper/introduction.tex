The importance of surveillance in our daily life has been constantly growing in the past couple of decades, and with that, also the need for more efficient and sophisticated mechanisms for surveillance. One of the major challenges comes from the need to use surveillance in large and complex environments, where it is not always feasible or cost effective to have complete surveillance coverage of the entire area at all times. Furthermore, sensors cannot necessarily classify threats and often require human intervention to assess the threat level. It can thus be necessary to deploy multiple mobile sensors, which working together, along with static sensors, must maintain a sufficient level of knowledge on the location of a potential threat. This is particularly crucial in applications where it is necessary to monitor a potential threat which can move over a large area until it can be addressed by security.

In a formal setting, designing a surveillance strategy for a (mobile) sensor network dealing with a potentially adversarial target can be modelled as a two-player game in which one player represents the sensor network and the other player represents the adversary. There are several variants of such games, including pursuit-evasion games~\cite{Chung2011} and graph-searching games~\cite{Kreutzer11}. In these games the problem is formulated as enforcing eventual detection, which is, in essence, a search problem –- once the target is detected, the game ends. This is too restrictive for applications where the goal is not capture, but instead maintain information over the adversary location for an unbounded time horizon.

Another class of games used in physical security are \emph{Stackelberg} games, also known as leader-follower games. In such games the defender acts first, for example by placing their defence system, and the attacker follows with his action, possibly after obtaining information about the placed defence system. In recent years Stackelberg games have seen use in, among others, LAX airport~\cite{Pita08},~\cite{jain2012overview} and the US Coast Guard~\cite{An11}. These games aim to compute randomized policies for the defender to protect target locations from an attacker. Extensions of this model~\cite{Basilico12} have been proposed to generate infinite-horizon patrolling strategies either for mobile resources alone or in concert with static alarm triggers~\cite{basilico2016security,Munoz13}. However, these models cannot be used to reason about the uncertain set of possible  locations of active threats. Our objective is not to just compute a patrolling strategy, but to quantify the sensor network's knowledge of the possible locations of active threats and use this to synthesize strategies for the mobile sensors that provide knowledge guarantees on the threat location over an infinite-time horizon.

As a motivating case study in this paper, we will consider the use of autonomous drones working with static sensors in wildlife conservation. Several native species are listed as critically endangered~\cite{mulero2014remotely}, and need to be protected from poaching. UAVs are increasingly being adopted for monitoring of illegal hunting and poaching ~\cite{schiffman2014drones}, though in almost all cases they are still remotely controlled ~\cite{mulero2014remotely}. In Kenya, for example, remotely controlled drones were deployed in 2014~\cite{Kenya} in an attempt to reduce poaching  by providing constant surveillance, allowing authorities to arrest rhino poachers when they are sensed by the drones. However, there has not been much use of autonomous UAVs as of yet, and proposed plans involve drones following pre-programmed paths~\cite{Koh12}. In this paper, we propose a method for automatically constructing  \emph{autonomous reactive surveillance} strategies for multiple mobile sensors (like UAVs) working in concert with static sensors in the field. 

We study the problem of synthesizing strategies for enforcing \emph{temporal surveillance objectives}, such as the requirement to never let the sensor network's uncertainty about the target's location exceed a given threshold, or recapturing the target every time it escapes. To this end, we consider surveillance objectives specified in linear temporal logic (LTL), equipped with basic surveillance predicates. Our computational model is that of a two-player game played on a finite graph, whose nodes represent the joint possible locations of all the mobile sensors and the target, and whose edges model the possible (deterministic) moves between locations. The mobile sensors play the game with partial information, as they can only observe the target when  it is in the area of sight of one of the sensors. The target, on the other hand, always has full information about the location of the entire sensor network. In that way, we consider a model with one-sided partial information, making the computed strategy for the agent robust against a potentially more powerful adversary. \looseness=-1

We formulate surveillance strategy synthesis as the problem of computing a joint winning strategy for the multiple mobile sensors in a partial-information game with a surveillance objective. Partial-information games with LTL objectives have been well studied~\cite{DoyenR11,Chatterjee2013} and it is well known that even for very simple objectives the synthesis problem is EXPTIME-hard~\cite{Reif84,BerwangerD08}. One proposed method to overcome this is to use abstraction in order to reduce the number of belief states in the game \cite{Bhar16}. The price of this approach is that it overapproximates the set of possible target locations (that is, makes the sensors' knowledge less precise) which can make satisfying stricter surveillance requirements difficult. There is thus a trade-off between strictness of surveillance requirements, i.e, how closely a target needs to be tracked, and the size of the abstract game necessary for a surveillance strategy to exist. 

\Rayna{The first sentence of this paragraph makes no sense, since all of the preceding paragraphs already talk about multiple sensors.}
\Suda{My goal is here is to show the benefit of multiple sensors from a technical standpoint in surveillance games, while previous paragraphs were a high level motivation. The explicit benefit is that stricter surveillance requirements can be satisfied with a smaller number of states in the abstract game. }
Multiple sensors working together along with static sensors can observe more than a single sensor and can work jointly to satisfy stricter surveillance requirements even with an overestimated belief - which will greatly reduce the size of the abstract game. However, the size of the state space is exponential in the number of sensors (assuming all sensors have the same set of possible locations). To address this, we propose a \emph{decentralized} synthesis method that aims to compute a surveillance strategy for each mobile sensor separately. We do this by decomposing the original surveillance game into a subgame for each sensor. The global surveillance objective is then broken up into a local objective for each subgame which we can solve using off-the-shelf reactive synthesis tools. We then guarantee
with the use of static sensors, that each mobile sensor satisfying its local surveillance objective will guarantee satisfaction of the global surveillance objective.

Synthesis from LTL specifications~\cite{Pnueli1989}, especially from formulae in the efficient GR(1) fragment~\cite{Piterman2006}, has been extensively used in robotic planning (e.g.~\cite{wong2012,Kress2007}), but surveillance-type objectives, such as the ones we study here, have not been considered so far. Epistemic logic specifications~\cite{MeydenV98} can refer to the knowledge of the agent about the truth-value of logical formulas, but, contrary to our surveillance specifications, are not capable of expressing requirements on the size of the agent's uncertainty. There has also been work in decentralized synthesis for GR(1) specifications, however, the synthesis process often involves a centralized computation as in \cite{Kloetzer06} or synchronization \cite{Salar17,Kloetzer11}. In our decentralized approach, the sensors require no communication as simply satisfying their local properties guarantees global satisfaction.



