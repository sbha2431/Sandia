In large or complex environments, it is not always feasible or cost effective to have complete surveillance coverage of the entire area at all times. Furthermore, sensors cannot necessarily classify threats and often require human intervention to assess the threat level. It can thus be necessary for multiple mobile sensors to work together to maintain a level of knowledge on the location of a potential threat. This is particularly crucial in applications where it is necessary to monitor a potential threat which can move over large areas until it can be dealt with.  

Dealing with a potentially adversarial player, the standard approach is to model the problem as a two-player non-cooperative game where one player represents the sensor network and the other represents the adversary. These \emph{security games} are often modeled as \emph{Stackenberg} games, otherwise known as leader-follower games with the defender as the leader and attacker as the follower. There are several variants of such games, including pursuit-evasion games~\cite{Chung2011} and graph-searching games~\cite{Kreutzer11}. In these games the problem is formulated as enforcing eventual detection, which is, in essence, a search problem – once the target is detected, the game ends. This is too restrictive for applications where the goal is not capture, but instead maintain information over the adversary location for an unbounded time horizon.

One class of Stackenberg games have seen use in, among others, LAX airport~\cite{Pita08},~\cite{jain2012overview} and the US Coast Guard~\cite{An11}. These games aim to compute randomized policies for the defender to protect target locations from an attacker. Extensions to this model~\cite{Basilico12} have been proposed to generate infinite-horizon patrolling strategies even in concert with alarm triggers~\cite{basilico2016security,Munoz13}. However, these models cannot be used to reason over the belief on the location of active threats. Our objective is not to compute a patrolling strategy, but to quantify the belief of the sensor network on the location of active threats and use this to synthesize strategies for the mobile sensors that provide knowledge guarantees on the threat location over an infinite-time horizon.

As a motivating case study in this paper, we will consider the use of autonomous drones in wildlife conservation. The two species of African rhinoceros, the black and white rhinoceros, are highly vulnerable, with the black rhinoceros listed as critically endangered ~\cite{mulero2014remotely} and need to be protected from poaching. UAVs are increasingly being adopted for monitoring of illegal hunting and poaching ~\cite{schiffman2014drones}, though in many cases they are still remotely controlled ~\cite{mulero2014remotely}. In Kenya, for example, remotely controlled drones were deployed in 2014~\cite{Kenya} in an attempt to reduce poaching  by providing constant surveillance, allowing authorities to arrest rhino poachers when they are sensed by the drones. However, there has not been much use of autonomous UAVs as of yet, and proposed plans involve drones following pre-programmed paths~\cite{Koh12}. In this paper, we propose a \emph{distributed reactive synthesis} approach to the autonomous distributed surveillance problem. 

We study the problem of synthesizing strategies for enforcing \emph{temporal surveillance objectives}, such as the requirement to never let the sensor network's uncertainty about the target's location exceed a given threshold, or recapturing the target every time it escapes. To this end, we consider surveillance objectives specified in linear temporal logic (LTL), equipped with basic surveillance predicates. Our computational model is that of a two-player game played on a finite graph, whose nodes represent the joint possible locations of all the mobile sensors and the target, and whose edges model the possible (deterministic) moves between locations. The mobile sensors play the game with partial information, as they can only observe the target when  it is in the area of sight of one of the sensors. The target, on the other hand, always has full information about the location of the entire sensor network. In that way, we consider a model with one-sided partial information, making the computed strategy for the agent robust against a potentially more powerful adversary. \looseness=-1

We formulate surveillance strategy synthesis as the problem of computing a joint winning strategy for the multiple mobile sensors in a partial-information game with a surveillance objective. Partial-information games with LTL objectives have been well studied~\cite{DoyenR11,Chatterjee2013} and it is well known that even for very simple objectives the synthesis problem is EXPTIME-hard~\cite{Reif84,BerwangerD08}. One proposed method to overcome this is to construct an \emph{abstract belief game} to greatly reduce the number of belief states in the game \cite{Bhar16}. The price of this approach is that it overestimates the belief of the target location which can make satisfying stricter surveillance requirements difficult. There is thus a trade-off - if the surveillance requirement is strict, i.e, the target needs to be closely tracked, size of the abstract belief game may be need to be larger in order for the surveillance specification to be realizable. 

This trade-off can be mitigated by the addition of multiple sensors as they can observe more than a single sensor and can work together to satisfy surveillance requirements even with an overestimated belief. However, the size of the state space is exponential in the number of sensors (assuming all sensors have the same set of possible locations). To address this, we propose a \emph{decentralized} synthesis method that aims to compute a surveillance strategy for each sensor separately. We do this by decomposing the original surveillance game into a subgame for each sensor. The global surveillance objective is then broken up into a local objective for each subgame which we can solve using off-the-shelf reactive synthesis tools. We then show that, under some assumptions, i.e, the presence of some \emph{static} sensors, we can guarantee that the each sensor satisfying its local surveillance objective will guarantee satisfaction of the global surveillance objective.






