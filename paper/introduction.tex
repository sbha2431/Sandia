The importance of surveillance in our daily life has been constantly growing in the past couple of decades, and with that, also the need for more efficient and sophisticated mechanisms for surveillance. One of the major challenges comes from the need to perform surveillance in large and complex environments, where it is not always feasible or cost effective to have complete surveillance coverage of the entire area at all times. Furthermore, sensors  might not always be able to classify threats, and often require human intervention to assess the threat level. It can thus be necessary to deploy multiple mobile sensors, that work together with conventional static sensors to maintain a sufficient level of knowledge on the location of a potential threat. This is particularly crucial in applications where it is necessary to monitor a potential threat which can move over a large area until it can be addressed by security.

In a formal setting, designing a surveillance strategy for a (mobile) sensor network dealing with a potentially adversarial target can be modelled as a two-player game in which one player represents the sensor network and the other player represents the adversary. There are several variants of such games, including pursuit-evasion games~\cite{Chung2011} and graph-searching games~\cite{Kreutzer11}. In these games the problem is formulated as enforcing eventual detection, which is, in essence, a search problem –- once the target is detected, the game ends. These types of games are too restrictive for applications where the goal is not to capture, but instead  to maintain information about the location of the adversary for an unbounded time horizon.

Another class of games used in physical security are \emph{Stackelberg} games, also known as leader-follower games. In such games the defender acts first, for example by placing their defence system, and the attacker follows with his action, possibly after obtaining information about the placed defence system. In recent years Stackelberg games have seen use in, among others, LAX airport~\cite{Pita08},~\cite{jain2012overview} and the US Coast Guard~\cite{An11}. These games aim to compute randomized policies for the defender to protect target locations from an attacker. Extensions of this model~\cite{Basilico12} have been proposed to generate infinite-horizon patrolling strategies either for mobile resources alone or in concert with static alarm triggers~\cite{basilico2016security,Munoz13}. However, these models cannot be used to reason about the uncertain set of possible  locations of dynamic threats.

Our objective in this work  is not to just compute a patrolling strategy, but also to quantify the sensor network's knowledge of the possible locations of active threats and use this information to synthesize strategies for the mobile sensors that provide knowledge guarantees on the threat location over an infinite-time horizon.

As a motivating case study in this paper, we consider the use of autonomous drones working with in cooperation with static sensors in wildlife conservation. UAVs are increasingly being adopted for monitoring of illegal hunting and poaching~\cite{schiffman2014drones}, though they are mostly remotely controlled~\cite{mulero2014remotely}. In Kenya, for example, remotely controlled drones were deployed in 2014 in an attempt to reduce poaching  by providing constant surveillance~\cite{Kenya}, allowing authorities to arrest rhino poachers when they are sensed by the drones. Autonomous UAVs  have not been used in this setting yet, and proposed plans involve drones following pre-programmed paths~\cite{Koh12}. In this paper, we propose a method for automatically constructing  \emph{autonomous reactive surveillance} strategies for multiple mobile sensors (like UAVs) working in concert with static sensors in the field. 

We study the problem of synthesizing strategies for enforcing \emph{temporal surveillance objectives}, such as the requirement to never let the sensor network's uncertainty about the target's location exceed a given threshold, or recapturing the target every time it escapes. To this end, we consider surveillance objectives specified in linear temporal logic (LTL), equipped with basic surveillance predicates. Our computational model is that of a two-player game played on a finite graph, whose nodes represent the joint possible locations of all the mobile sensors and the target, and whose edges model the possible (deterministic) moves between locations. The mobile sensors play the game with partial information, as they can only observe the target when  it is in the area of sight of one of the sensors. The target, on the other hand, always has full information about the location of the entire sensor network. In that way, we consider a model with one-sided partial information, making the computed strategy for the agent robust against a potentially more powerful adversary. \looseness=-1

We formulate surveillance strategy synthesis as the problem of computing a joint winning strategy for the multiple mobile sensors in a partial-information game with a surveillance objective. Partial-information games with LTL objectives have been well studied~\cite{DoyenR11,Chatterjee2013} and it is well known that even for very simple objectives the synthesis problem is EXPTIME-hard~\cite{Reif84,BerwangerD08}. One proposed method to overcome this is to use abstraction in order to reduce the number of belief states in the game \cite{arxiv}. The price of this approach is that it overapproximates the set of possible target locations (that is, makes the sensors' knowledge less precise) which can make satisfying stricter surveillance requirements difficult. There is thus a trade-off between strictness of surveillance requirements, i.e, how closely a target needs to be tracked, and the size of the abstract game necessary for a surveillance strategy to exist. 

Sensor networks with a large number of dynamic sensors can achieve a better coverage, and thus, in general, require much smaller abstractions to satisfy a given surveillance objective. However, even when using abstraction, the size of the game is exponential in the number of sensors. To address the blow-up of the state space incurred by large number of sensors, we propose a \emph{decentralized} synthesis method that aims to compute a surveillance strategy for each mobile sensor separately. We do this by decomposing the original surveillance game into a set of subgames, one for each sensor. To this end, the global surveillance objective is  broken up into a local objective for each subgame. We can then solve each subgame separately, using off-the-shelf reactive synthesis tools. Our reduction guarantees  that if the surveillance strategy synthesized for each mobile sensor satisfies its local surveillance objective, then the composition of the strategies fulfils the global surveillance objective.

Synthesis from LTL specifications~\cite{Pnueli1989}, especially from formulae in the efficient GR(1) fragment~\cite{Piterman2006}, has been extensively used in robotic planning (e.g.~\cite{wong2012,Kress2007}), but surveillance-type objectives, such as the ones we study here, have not been considered so far. Epistemic logic specifications~\cite{MeydenV98} can refer to the knowledge of the agent about the truth-value of logical formulas, but, contrary to our surveillance specifications, are not capable of expressing requirements on the size of the agent's uncertainty. There has also been work in decentralized synthesis for GR(1) specifications, however, the synthesis process often involves a centralized computation as in \cite{Kloetzer06} or synchronization \cite{Salar17,Kloetzer11}. Our approach, on the other hand is fully  decentralized and the sensors require no communication as simply satisfying their local properties guarantees the global objective.. 

In a companion submission to CDC 2018 we describe a framework for formalizing \emph{single-agent} surveillance synthesis as a two-player game with partial information, and propose an abstraction-based method for solving such games. The interested reader is referred to the expended version~\cite{arxiv} for details about the abstraction-based synthesis method .



